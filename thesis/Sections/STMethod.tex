\documentclass[../thesis.tex]{subfiles}

\begin{document}
\chapter{A discontinuous Galerkin method for the instationary heat equation}
\label{sec:dG-method}
As a next step, we will introduce a discontinuous Galerkin method for the heat equation based on the work \cite{Neumueller}. Some expansion of his theory is required given that his work does not consider mixed boundary conditions, but rather assumes separate boundary conditions that would be either Neumann or Dirichlet boundary conditions.

We will keep the method as general as possible so that both, boundary and inner heat source formulations can be discretized. As seen during the derivation of the weak formulation of the state system, we require a certain smoothness of the domain, so in the following we assume that $\Omega \subset \R^d$, $d = 1, 2, 3$ is a bounded Lipschitz domain.
For its boundary we assume that $\partial \Omega = \bar{\Gamma}_D \cup \bar{\Gamma}_R$, $\Gamma_D \cap \Gamma_R = \emptyset$ holds, where $\Gamma_D$ and $\Gamma_R$ denote the part of the boundary with Dirichlet and Robin boundary conditions in effect, respectively.

The problem we consider is then
\begin{equation}
\label{eq:dg-model-prob}
\begin{IEEEeqnarraybox}[][c]{rCl"lCl}
\frac{\partial f}{\partial t} (x, t) - \lapl f(x, t) &=& g_I(x, t) & \text{for } (x, t) \in \Omega &\coloneqq& \Omega \times (0, T), \\
f(x, t) &=& 0 & \text{for } (x, t) \in \Sigma_D &\coloneqq& \Gamma_D \times (0, T) \\
n_x(x,t) \cdot \nabla_x f(x, t) + \alpha f(x, t) &=& g_R(x, t) & \text{for } (x, t) \in \Sigma_R &\coloneqq& \Gamma_R \times (0, T) \\
f(x, 0) &=& f_0(x) & \text{for } (x, t) \in \Omega.
\end{IEEEeqnarraybox}
\end{equation}
Note that by choosing $g_I$ and $g_R$ appropriately, we can cover both, inner and boundary heat sources.
\begin{remark}
One could directly extend this method to include a mixture of Neumann and Robin boundary conditions on different parts of the boundary. However as we will see, that setting is not useful for optimal control problems and by choosing $\alpha = 0$, we already cover Neumann boundary conditions as discussed in \cite{Neumueller}. Therefore, we do not cover this setting here, in order to simplify the notation.
\end{remark}

Following the proceedings of \cite[Chapter 2]{Neumueller}, we can derive a space-time discontinuous Galerkin method for this problem.

In order to arrive at a discretization, we make an approach of decomposing the space-time cylinder $Q$ into $N \in \mathbb{N}$ finite elements, whose collection we denote by $\meshT_N$. More precisely we work with simplexes of mesh size $h_\ell = \sqrt[d+1]{|\tau_\ell|}$ and assume the space-time domain $Q$ can be decomposed into simplexes without error, i.e.
\[
	\bar{Q} = \bar{\meshT}_N \coloneqq \bigcup_{\ell=1}^N \bar{\tau}_\ell.
\]
By this definition, $Q$ is a polygonally-bounded domain.
For two neighboring elements $\tau_k, \tau_\ell \in \meshT_N$, we define the interior facet $\Gamma_{k \ell}$ as
\[
	\Gamma_{k \ell} \coloneqq \bar{\tau}_k \cap \bar{\tau}_\ell
\]
if the set $\Gamma_{k \ell}$ is a $d$-dimensional manifold, that is of dimensional one smaller than the dimension of the space-time cylinder $Q$. The set of all interior facets of the decomposition $\meshT_N$ will be denoted by $\intfI_N$.

The general idea for the discontinuous Galerkin formulation we're going to use is to use a symmetric interior penalty approach for the term $\lapl f$ and an upstream bilinear form for $\partial_t f$.
For both approaches, local continuity on the elements of the mesh $\meshT_N$ is going to be assumed, while permitting discontinuities on the interfaces. In order to make up for this, jumps across the interfaces are going to be punished using penalty terms.
Given that we're working with two very different terms, one operating only as space derivative, and the other only operating as time derivative, we have to punish jumps in time and space directions differently.
Hence, we have to introduce some more notations before being able to proceed with the discretization itself.

For an element $\tau_k$ we denote the outer unit normal vector on an interface $\Gamma_{k \ell} \in \intfI_N$ as
\[
	\markvec{n_k} = \left( \markvec{n_{k, x}}, n_{k, t} \right)^\tp.
\]
Consequently, $\markvec{n_\ell} = - \markvec{n_k}$ on that interface.
If we have more than one space dimension $\markvec{n_{k, x}}$ will be a vector, whereas $n_{k, t}$ will always be a scalar value.

In order to introduce discontinuous Galerkin bilinear forms, we require notations for jumps and averages across interfaces.
Therefore, we introduce the following notations for a scalar function $v$ on an interface $\Gamma_{k \ell}$:
\begin{itemize}
\item The jump in space direction will be denoted by
\[
	\ljump v \rjump_{\Gamma_{k \ell}, x} (x, t) \coloneqq \left. v \right|_{\tau_k} (x, t) \cdot \markvec{n_{k, x}} + \left. v \right|_{\tau_\ell} (x, t) \cdot \markvec{n_{\ell, x}} \quad \text{for } (x, t) \in \Gamma_{k, \ell}.
\]
\item For the jump in time direction we write
\[
	\ljump v \rjump_{\Gamma_{k \ell}, t} (x, t) \coloneqq \left. v \right|_{\tau_k} (x, t) \cdot n_{k, t} + \left. v \right|_{\tau_\ell} (x, t) \cdot  n_{\ell, t} \quad \text{for } (x, t) \in \Gamma_{k, \ell}.
\]
\item The average of $v$ on the interface is going to be denoted as
\[
	\lavg v \ravg_{\Gamma_{k \ell}} (x, t) \coloneqq \frac{1}{2} \left[ v|_{\tau_k} (x, t) + v|_{\tau_\ell}(x, t) \right] \quad \text{for } (x, t) \in \Gamma_{k, \ell}.
\]
\item The so called upwind in time direction is defined by
\[
	\lupw v \rupw_{\Gamma_{k \ell}} (x, t) \coloneqq \begin{cases}
	v|_{\tau_k}(x, t) & \text{for } n_{k, t} > 0, \\
	0 & \text{for } n_{k, t} = 0,\\
	v|_{\tau_\ell}(x, t) & \text{for } n_{k, t} < 0,
	\end{cases} \quad \text{for } (x, t) \in \Gamma_{k \ell}.
\]
\end{itemize}
Note that with these definitions, the jump in space direction is a vector itself, whereas the jump in time direction is a scalar value.

With these notations introduced, we can proceed. As already remarked, we're looking for an overall discontinuous approach while maintaining elementwise continuity. As of such, we simply demand for our ansatz space that the functions in it restricted to every element are polynomial of order $p$, where $p$ can be freely chosen.
In precise terms, we obtain an ansatz space of
\[
	\Shp(\mathcal{T}_N) \coloneqq \left\{ v_h \in L_2(Q) \gmid v_h |_{\tau_\ell} \in \polyP_p(\tau_\ell) \text{ for all } \tau_\ell \in \mathcal{T}_N \text{ and } v_h = 0 \text{ on } \Sigma_D \right\}.
\]
As it is common practice, we demand zero values on the Dirichlet boundary.
In order to obtain a discontinuous Galerkin formulation in this space, we multiply the differential equation with a test function $v$ of this space and integrate over the space-time cylinder:
\[
	\iint_Q \partial_t f(x, t) v(x, t) \dd x \dd t - \iint_Q \lapl f(x, t) v(x, t) \dd x \dd t = \iint_Q g_I(x, t) v(x, t) \dd x \dd t.
\]
We begin by working with the second term containing the Laplace operator.
As before, we apply Green's first identity to deal with it and make use of the given boundary conditions
\begin{IEEEeqnarray*}{rCl}
	- \iint_Q \lapl f v \dd x \dd t &=& \iint_Q \nabla f \cdot \nabla v \dd x \dd t - \iint_{\Sigma} (\nabla f v) \cdot \nu \dd s(x) \dd t \\
	&=& \iint_Q \nabla f \cdot \nabla v \dd x \dd t - \iint_{\Sigma_R} \left(-\alpha f + g_R \right) v \dd s(x) \dd t \\
	&=& \iint_Q \nabla f \cdot \nabla v \dd x \dd t + \alpha \iint_{\Sigma_R} f v \dd s(x) \dd t - \iint_{\Sigma_R} g_R v \dd s(x) \dd t
\end{IEEEeqnarray*}
Here, we used that $v$ is zero on the Dirichlet boundary, so that the term vanishes.
Moreover, the last term can be added to the right hand side of the equation system.

In order to obtain an approach for the remaining term,
\[
	\iint_Q \nabla u \nabla v \dd x \dd t
\]
we use the so called ``symmetric interior penalty'' method, see \cite[Chapter 4.2]{DiPietroErn} for the derivation of the method and \cite[(4.12), p.\ 125]{DiPietroErn} for the final result\footnote{In comparison to \cite[(4.12), p.\ 125]{DiPietroErn} the bilinear form looks slightly different, as we use a different definition of jumps, compare \cite[Remark 1.20, p.\ 12]{DiPietroErn}.}:
\begin{IEEEeqnarray*}{rCl}
	a^\sip(f_h, v_h) &\coloneqq& \sum_{\ell = 1}^N \iint_{\tau_\ell} \nabla_x f_h(x, t) \cdot \nabla_x v_h(x, t) \dd x \dd t \\
	&& {} - \sum_{\Gamma_{k\ell} \in \intfI_N} \iint_{\Gamma_{k \ell}} \lavg \nabla_x f_h \ravg_{\Gamma_{k \ell}} (x, t) \ljump v_h \rjump_{\Gamma_{k \ell}, x} (x, t) \dd s(x) \dd t \\
	&& {} - \sum_{\Gamma_{k\ell} \in \intfI_N} \iint_{\Gamma_{k \ell}} \ljump f_h \rjump_{\Gamma_{k \ell}, x} (x, t) \lavg \nabla_x v_h \ravg_{\Gamma_{k \ell}} (x, t) \dd s(x) \dd t \\
	&& {} + \sum_{\Gamma_{k \ell} \in \intfI_N} \frac{\sigma}{\bar{h}_{k\ell}} \iint_{\Gamma_{k \ell}} \ljump f_h \rjump_{\Gamma_{k \ell}, x} \cdot \ljump v_h \rjump_{\Gamma_{k \ell}, x} \dd s(x) \dd t.
\end{IEEEeqnarray*}
Here, $\bar{h}_{k \ell}$ is a local length scale associated with the interior facet $\Gamma_{k \ell}$ and $\sigma > 0$ is a stability parameter. For $\bar{h}_{k \ell}$ there are several choices possible, see \cite[Remark 4.6, p.\ 125]{DiPietroErn}. We will use the average of the longest edges of the elements $\tau_k$ and $\tau_\ell$, denoted by $h_k$ and $h_\ell$, respectively:
\[
	\bar{h}_{k \ell} = \lavg h \ravg_{\Gamma_{k \ell}} \coloneqq \frac{1}{2} \left( h_k + h_\ell \right).
\]
The requirements that the choice of $\sigma > 0$ has to fulfill become obvious while proving stability of the bilinear form.

For the term stemming from the Robin boundary we define:
\begin{equation}
\label{eq:aR-definition}
	a^R(f_h, v_h) \coloneqq \alpha \iint_{\Sigma_R} f_h v_h \dd s(x) \dd t.
\end{equation}
With these definitions, we set
\[
	a(f_h, v_h) = a^\sip(f_h, v_h) + a^R(f_h, v_h).
\]
We still need to discretize the remaining term,
\[
	\iint_Q \frac{\partial f}{\partial t} v \dd x \dd t.
\]
As we have differentiability inside single elements of $\meshT_N$, we proceed element wise and use integration by parts:
\begin{IEEEeqnarray*}{rCl}
	\iint_Q \frac{\partial f}{\partial t}(x, t) v(x, t) \dd x \dd t &=& \sum_{\ell = 1}^N \iint_{\tau_\ell} \frac{\partial f}{\partial t}(x, t) v(x, t) \dd x \dd t \\
	&=& - \sum_{\ell = 1}^N \bigg( \iint_{\tau_\ell} \frac{\partial v}{\partial t}(x, t) f(x, t) \dd x \dd t \\
	&& {} \qquad\quad {} + \iint_{\partial \tau_\ell} v(x, t) f(x, t) \cdot n_{k, t} \dd s(x) \dd t \bigg) \\
	&=& - \sum_{\ell = 1}^N \iint_{\tau_\ell} \frac{\partial v}{\partial t}(x, t) f(x, t) \dd x \dd t \\
	&& {} + \sum_{\Gamma_{k \ell} \in \intfI_N} \iint_{\Gamma_{k \ell}} \ljump f (x, t) v(x, t) \rjump_{\Gamma_{k \ell}, t} \dd s(x) \dd t \\
	&& {} + \int_{\Omega} f(x, T) v(x, T) \dd s(x) - \int_{\Omega} f(x, 0) v(x, 0) \dd s(x)
\end{IEEEeqnarray*}
Given the initial condition $f(\cdot, 0) = f_0(\cdot)$, we can add the term
\[
	\int_{\Omega} f(x, 0) v(x, 0) \dd s(x) = \int_{\Omega} f_0(x) v(x, 0) \dd s(x)
\]
to the right-hand side. For continuous $f$ and $v$, the term
\[
	\iint_{\Gamma_{k \ell}} \ljump f(x, t) v(x, t) \rjump_{\Gamma_{k \ell}, t} \dd s(x) \dd t
\]
will be zero. For our discontinuous approach however it is not necessary going to vanish.
Considering we aim for using an upstream approach, i.e.\ using values in direction of the derivative.
Thus, we use that for a \textit{continuous} function $f$ we would have the following equality on any interface $\Gamma_{k \ell} \in \intfI_N$:
\begin{IEEEeqnarray*}{rCl}
	\iint_{\Gamma_{k \ell}} \ljump f (x, t) v(x, t) \rjump_{\Gamma_{k \ell}, t} \dd s(x) \dd t &=& \iint_{\Gamma_{k \ell}} f (x, t) \ljump v \rjump_{\Gamma_{k \ell}, t} (x, t) \dd s(x) \dd t \\
	&=& \iint_{\Gamma_{k \ell}} \lupw f \rupw_{\Gamma_{kl}} (x, t) \ljump v \rjump_{\Gamma_{k \ell}, t} (x, t) \dd s(x) \dd t.
\end{IEEEeqnarray*}
For elements of the \textit{discrete} function spaces this equality will not hold, but it will penalize jumps in a way we desire.
Overall, obtain a second bilinear form:
\begin{IEEEeqnarray*}{rCl}
	b(f_h, v_h) &\coloneqq& - \sum_{\ell = 1}^N \iint_{\tau_\ell} f_h(x, t) \frac{\partial v_h}{\partial t}(x, t) \dd x \dd t + \int_{\Omega} f_h(x, T) v_h(x, T) \dd s(x) \\
	&& {} + \sum_{\Gamma_{k\ell} \in \mathcal{I}_N} \iint_{\Gamma_{k \ell}} \lupw f_h \rupw_{\Gamma_{k \ell}} (x, t) \ljump v_h \rjump_{\Gamma_{k \ell}, t} (x, t) \dd s(x) \dd t.
\end{IEEEeqnarray*}
Before moving on, we remark that if we integrate by parts, we obtain:
\begin{equation}
\label{eq:b-up-down-equal}
\begin{IEEEeqnarraybox}[][c]{rCl}
b(f_h, v_h) &=& - \sum_{\ell = 1}^N \iint_{\tau_\ell} f_h(x, t) \frac{\partial v_h}{\partial t}(x, t) \dd x \dd t + \int_{\Omega} f_h(x, T) v_h(x, T) \dd s(x) \\
&& {} + \sum_{\Gamma_{k\ell} \in \mathcal{I}_N} \iint_{\Gamma_{k \ell}} \lupw f_h \rupw_{\Gamma_{k \ell}} (x, t)\ljump v_h \rjump_{\Gamma_{k \ell}, t} (x, t) \dd s(x) \dd t . \\
&=& \sum_{\ell = 1}^N \iint_{\tau_\ell} \frac{\partial f_h}{\partial t}(x, t) v_h(x, t) \dd x \dd t + \int_{\Omega} f_h(x, 0) v_h(x, 0) \dd s(x) \\
&& {} - \sum_{\Gamma_{k \ell} \in \mathcal{I}_N} \iint_{\Gamma_{k \ell}} \ljump f_h \rjump_{\Gamma_{k \ell}, t}(x, t) \ldwnd v_h \rdwnd_{\Gamma_{k \ell}}(x, t) \dd s(x) \dd t
\end{IEEEeqnarraybox}
\end{equation}
for all $f_h, v_h \in \Shp(\mathcal{T}_N)$, where the downwind in time direction on an interior facet $\Gamma_{k \ell} \in \intfI_N$ is given by
\[
	\ldwnd v \rdwnd_{\Gamma_{k \ell}} (x, t) \coloneqq \begin{cases}
	v|_{\tau_\ell}(x, t) & \text{for } n_{k, t} > 0, \\
	0 & \text{for } n_{k, t} = 0,\\
	v|_{\tau_k}(x, t) & \text{for } n_{k, t} < 0,
	\end{cases} \quad \text{for } (x, t) \in \Gamma_{k \ell}.
\]
In order to see this equivalence, we apply integration by parts
\begin{IEEEeqnarray*}{rCl}
	- \sum_{\ell = 1}^{N} \iint_{\tau_\ell} f_h(x, t) \frac{\partial v}{\partial t}(x, t) \dd s(x) \dd t &=& \sum_{\ell = 1}^{N} \iint_{\tau_\ell} \frac{\partial f}{\partial t}(x, t) v_h(x, t) \dd s(x) \dd t \\
	&& {} + \int_{\Omega} u_h(x, T) v_h(x, T) \dd s(x) \\
	&& {} - \int_{\Omega} u_h(x, 0) v_h(x, 0) \dd s(x) \\
	&& {} + \sum_{\Gamma_{k \ell} \in \mathcal{I}_N} \iint_{\Gamma_{k \ell}} \ljump u_h(x, t) v_h(x, t) \cdot n_{k, t} \rjump_{\Gamma_{k \ell}, t} \dd s(x) \dd t 
\end{IEEEeqnarray*}
By definition $\lupw n_{k, t} \rupw_{\Gamma_{k \ell}}$ is the value of the time component of the outer normal on the upstream element and $\lupw n_{k, t} \rupw_{\Gamma_{k \ell}} = - \ldwnd n_{k, t} \rdwnd_{\Gamma_{k \ell}}$ holds. 
\begin{IEEEeqnarray*}{l}
	\sum_{\Gamma_{k \ell} \in \mathcal{I}_N} \iint_{\Gamma_{k \ell}} \ljump u_h(x, t) v_h(x, t) \cdot n_{k, t} \rjump_{\Gamma_{k \ell}, t} \dd s(x) \dd t \\
	\IEEEeqnarraymulticol{1}{r}{ \qquad\qquad {} = \sum_{\Gamma_{k \ell} \in \mathcal{I}_N} \iint_{\Gamma_{k \ell}} \left( \lupw f_h \rupw_{\Gamma_{k \ell}} \lupw v_h \rupw_{\Gamma_{k \ell}} - \ldwnd f_h \rdwnd_{\Gamma_{k \ell}} \ldwnd v_h \rdwnd_{\Gamma_{k \ell}} \right) \lupw n_{k, t} \rupw_{\Gamma_{k \ell}} \dd s(x) \dd t}
\end{IEEEeqnarray*}
By definition, one sees that for any function $v$ we have:
\begin{IEEEeqnarray*}{rCl}
	\ljump v \rjump_{\Gamma_{k \ell}, t}(x, t)  &=& \left. v \right|_{\tau_k} (x, t) \cdot  n_{k, t} + \left. v \right|_{\tau_\ell} (x, t) \cdot  n_{\ell, t} \\
	&=& \left( \lupw v \rupw_{\Gamma_{k\ell}}(x, t) - \ldwnd v \rdwnd_{\Gamma_{k\ell}} (x, t) \right) \lupw n_{k, t} \rupw_{\Gamma_{k \ell}}
\end{IEEEeqnarray*}
As of such:
\begin{IEEEeqnarray*}{l}
	\lupw f_h \rupw_{\Gamma_{k\ell}} \ljump v_h \rjump_{\Gamma_{k \ell}, t} - \left( \lupw f_h \rupw_{\Gamma_{k \ell}} \lupw v_h \rupw_{\Gamma_{k \ell}} - \ldwnd f_h \rdwnd_{\Gamma_{k \ell}} \ldwnd v_h \rdwnd_{\Gamma_{k \ell}} \right) \lupw n_{k, t} \rupw_{\Gamma_{k \ell}} \\
	\qquad {} = - \left( \lupw f_h \rupw_{\Gamma_{k\ell}} - \ldwnd f_h \rdwnd_{\Gamma_{k\ell}} \right) \lupw n_{k, t} \rupw_{\Gamma_{k \ell}} \ldwnd v_h \rdwnd_{\Gamma_{k \ell}} \\
	\qquad {} = - \ljump f_h \rjump_{\Gamma_{k\ell}, t} \ldwnd v_h \rdwnd_{\Gamma_{k \ell}}
\end{IEEEeqnarray*}
By adding up using this equation, we have shown that the above two forms of the bilinear form $b(\cdot, \cdot)$ are equivalent.

With this work done, we define the bilinear form $A(\cdot, \cdot)$ as
\[
	A(f_h, v_h) \coloneqq b(f_h, v_h) + a(f_h, v_h).
\]
Then we have a right hand side of
\[
	\tilde{g}(v_h) \coloneqq \iint_Q g_I v_h \dd x \dd t + \int_\Omega f_0(x) v_h(x, 0) \dd x + \int_{\Sigma_R} g_R v_h \dd s(x) \dd t 
\]
The problem formulation we obtain is then
\begin{equation}
\label{eq:dg-discrete-form}
	A(f_h, v_h) = \tilde{g}(v_h).
\end{equation}
Taking the exact solution $f$ of \cref{eq:dg-model-prob}, one can see
\begin{equation}
\label{eq:dg-Galerkin-orthogonality}
	A(f - f_h, v_h) = 0 \quad \text{for all } v_h \in \Shp(\meshT_N).
\end{equation}
This can be verified to hold by considering that $f$ is continuous and therefore all interface penalty terms vanish. Using partial integration for the remaining term of $a^\sip(f, v_h)$ (and for $b(f, v_h)$, which is equivalent to taking the steps we performed in reverse order), one can see right away that $A(f, v_h)$ and $A(f_h, v_h)$ yield the same right hand side.

Using a basis $\varphi_j$, $j = 1, \ldots, m$ of $\Shp$, i.e.
\[
	\Shp(\mathcal{T}_N) = \spn \{ \varphi_j \}_{j=1}^m, \quad f_h(x, t) = \sum_{j=1}^m \boldsymbol{f}[j] \varphi_j(x, t) \quad \text{for } f_h \in \Shp(\mathcal{T}_N),
\]
we obtain a formulation as a linear system:
\begin{equation}
\label{eq:dg-discrete-prob}
	A_h \boldsymbol{f} = \boldsymbol{g}
\end{equation}
with
\[
	A_h[i, j] \coloneqq A(\varphi_j, \varphi_i) \quad \text{and} \quad \boldsymbol{g}[i] \coloneqq \tilde{g}(\varphi_i). 
\]
\section{Adjoined problem treatment}
\label{sec:adj-dG-treatment}
We can additionally transport the same approach to obtain a discretization of $- \partial_t f$. For this we consider the problem
\begin{equation}
\label{eq:dg-adjoint-prob}
\begin{IEEEeqnarraybox}[][c]{rCl"lCl}
- \frac{\partial f}{\partial t} (x, t) - \lapl f(x, t) &=& g_I(x, t) & \text{for } (x, t) \in \Omega &\coloneqq& \Omega \times (0, T), \\
f(x, t) &=& 0 & \text{for } (x, t) \in \Sigma_D &\coloneqq& \Gamma_D \times (0, T) \\
n_x(x,t) \cdot \nabla_x f(x, t) + \alpha f(x, t) &=& g_R(x, t) & \text{for } (x, t) \in \Sigma_R &\coloneqq& \Gamma_R \times (0, T) \\
f(x, T) &=& f_T(x) & \text{for } (x, t) \in \Omega.
\end{IEEEeqnarraybox}
\end{equation}
It will later on enable us to work without the transformation $\tilde{p}(x, t) = p(x, T - t)$ when treating the adjoint state equation.
For this purpose, let us start with the term
\[
	\iint_Q - \frac{\partial f}{\partial t} v \dd x \dd t.
\]
Applying the same testing procedure as before, we obtain:
\begin{IEEEeqnarray*}{rCl}
	\int_Q - \frac{\partial f}{\partial t}(x, t) v(x, t) \dd x \dd t &=& \sum_{\ell = 1}^N \iint_{\tau_\ell} f(x, t) \frac{\partial v}{\partial t}(x, t) \dd x \dd t  \\
	&& {} - \iint_{\Gamma_{k \ell}} \ljump f(x, t) v(x, t) \rjump_{\Gamma_{k \ell}, t} \dd s(x) \dd t \\
	&& {} - \int_{\Omega} f(x, T) v(x, T) \dd s(x) + \int_{\Omega} u(x, 0) v(x, 0) \dd s(x) \dd t .
\end{IEEEeqnarray*}
In this case we need to impose boundary conditions at the end time $T$ instead of the start time $0$, i.e.
\[
	\int_{\Omega} f(x, T) v(x, T) \dd s(x) = \int_{\Omega} f_T(x) v(x, T) \dd s(x).
\]
This term can be moved to the right hand side, as before.

Given the derivative is now going against the time, it is reasonable to demand a propagation of information via downstream, instead of via upstream. The choice of the upstream was arbitrary anyways, as for continuous $u$, the up- and downstream values are going to be the same.
Thus, we make an alternative approach $b'$ as follows:
\begin{equation}
\label{eq:b-prime}
\begin{IEEEeqnarraybox}{rCl}
	b'(f_h, v_h) &\coloneqq& \sum_{\ell = 1}^N \iint_{\tau_\ell} f_h(x, t) \frac{\partial v_h}{\partial t}(x, t) \dd x \dd t + \int_{\Omega} f_h(x, 0) v_h(x, 0) \dd s(x) \\
	&& {} - \sum_{\Gamma_{k\ell} \in \mathcal{I}_N} \iint_{\Gamma_{k \ell}} \ldwnd f_h \rdwnd_{\Gamma_{k \ell}} (x, t)\ljump v_h \rjump_{\Gamma_{k \ell}, t} (x, t) \dd s(x) \dd t.
\end{IEEEeqnarraybox}
\end{equation}
The reason why we're giving this formulation is that we can see it to be equivalent to the original $b(\cdot, \cdot)$ in the following sense: For a $f_h$, let $\tilde{f}_h(x, t) = f_h(x, T - t)$ and let $Q'$ and $\meshT_N'$ be the result of transforming $Q$ and $\meshT_N$ via the same transformation $(x, t) \mapsto (x, T - t)$. Then:
 \begin{IEEEeqnarray*}{rCl}
	b'(f_h, v_h) &\coloneqq& \sum_{\ell = 1}^N \iint_{\tau_\ell} f_h(x, t) \frac{\partial v_h}{\partial t}(x, t) \dd x \dd t + \int_{\Omega} f_h(x, 0) v_h(x, 0) \dd s(x) \\
	&& {} - \sum_{\Gamma_{k\ell} \in \mathcal{I}_N} \iint_{\Gamma_{k \ell}} \ldwnd f_h \rdwnd_{\Gamma_{k \ell}} (x, t)\ljump v_h \rjump_{\Gamma_{k \ell}, t} (x, t) \dd s(x) \dd t \\
	&=& - \sum_{\ell = 1}^N \iint_{\tau_\ell} \tilde{f}_h(x, T - t) \frac{\partial \tilde{v}_h}{\partial t}(x, T - t) \dd x \dd t + \int_{\Omega} \tilde{f}_h(x, T) \tilde{v}_h(x, T) \dd s(x) \\
	&& {} + \sum_{\Gamma_{k\ell} \in \mathcal{I}_N} \iint_{\Gamma_{k \ell}} \ldwnd \tilde{f}_h \rdwnd_{\Gamma_{k \ell}} (x, T - t) \ljump \tilde{v}_h \rjump_{\Gamma_{k \ell}, t} (x, T - t) \dd s(x) \dd t \\
	&=& \sum_{\ell = 1}^N \iint_{\tau_\ell'} \tilde{f}_h(x, t) \frac{\partial \tilde{v}_h}{\partial t}(x, t) \dd x \dd t + \int_{\Omega} \tilde{f}_h(x, T) \tilde{v}_h(x, T) \dd s(x) \\
	&& {} - \sum_{\Gamma_{k\ell}' \in \mathcal{I}_N'} \iint_{\Gamma_{k \ell}'} \lupw \tilde{f}_h \rupw_{\Gamma_{k \ell}'} (x, t) \ljump \tilde{v}_h \rjump_{\Gamma_{k \ell}', t} (x, t) \dd s(x) \dd t.
\end{IEEEeqnarray*}
Note that for a given $\Gamma_{k\ell} \in \mathcal{I}_N$, the associated $\Gamma_{k\ell}'$ after the transformation possesses the outer normal vector normal vector with respect to its element $\tau_k'$
\[
	n_k' = (n_{k,x}, -n_{k,t})^\tp
\]
Hence, by definition this translates into $\ldwnd \tilde{f}_h \rdwnd_{\Gamma_{k \ell}}(x, T - t) = \lupw \tilde{f}_h \rupw_{\Gamma_{k \ell}'}(x, t)$.
We conclude therefore, that using $b'$ on the problem for $f_h$ is the same as using $b$ on the transformed geometry for $\tilde{f}_h$. Given that the mesh was an arbitrary simplicial decomposition, this does not restrict us in any way whatsoever.
However, if we can prove that using $b$ yields a working method - which we will do in the next chapter - then so does $b'$ for the formulation with $- \partial_t f$.

A closer examination reveals even one more thing:
\[
	b'(f_h, v_h) = b(v_h, f_h).
\]
In fact, we have already proven this statement. The downstream formulation of $b$, \cref{eq:b-up-down-equal}, compared with the definition of $b'$, \cref{eq:b-prime} is the same with the roles of $v_h$ and $f_h$ interchanged.

Noting that $a(\cdot, \cdot)$ was symmetric anyways, we can define the discontinuous Galerkin method using the bilinear form $A(v_h, f_h)$ instead of $A(f_h, v_h)$.
The corresponding right hand side is then
\[
	\tilde{g}'(v_h) \coloneqq \iint_Q g_I v_h \dd x \dd t + \int_\Omega f_T(x) v_h(x, T) \dd x + \int_{\Sigma_R} g_R v_h \dd s(x) \dd t 
\]
we obtain a formulation as a linear system:
\[
	A_h^\tp \boldsymbol{u} = \boldsymbol{g}'
\]
with
\[
	\boldsymbol{g}'[i] \coloneqq \tilde{g}'(\varphi_i). 
\]
\end{document}