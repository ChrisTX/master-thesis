\documentclass[../thesis.tex]{subfiles}

\begin{document}
\chapter{Numerical analysis of the discontinuous Galerkin method}
In comparison to \cite{Neumueller}, we require a theory suitable for Robin boundary conditions.
Overall, we want to use the following abstract existence result to prove existence of a solution.
\begin{theorem}
\label{thm:Necas-Babuska}
Let $U$ and $V$ be Hilbert spaces. A linear map $L : U \to V^\adj$ is an isomorphism if and only if $a : U \times V \to \R$ fulfills the following conditions:
\begin{enumerate}
\item Continuity. Let with $C > 0$:
\[
	| a(u, v) | \leq c_b \| u \|_U \| v \|_V.
\]
\item $\inf$-$\sup$ condition. Let with $c_e > 0$:
\[
	\sup_{v \in V} \frac{a(u, v)}{\| v \|_V} \geq c_e \| u \|_U \quad \text{for } u \in U.
\] 
\item For each $v \in V$, $v \neq 0$, there exists an element $u \in U$ with
\[
	a(u, v) \neq 0.
\]
\end{enumerate}
\begin{proof}
See \cite[3.6 Satz]{Braess}.
\end{proof}
\end{theorem}
\needspace{10\baselineskip}
Before we begin, certain assumptions in regards to the mesh $\meshT_N$ have to be made in order to prove solvability.
\begin{assumption}
\label{as:mesh-assumptions}
We assume that there are two constants $c_{R_1}, c_{R_2} > 0$ such that
\[
	c_{R_1} h_\ell^d \leq |\partial \tau_\ell| \leq c_{R_2} h_\ell^d \quad \forall \tau_\ell \meshT_N.
\]
Note that by definition $|\tau_\ell| = h_\ell^{d+1}$ for all $\tau_\ell \in \meshT_N$.
Moreover, let there be a constant $\tilde{c}_G \geq 1$ such that for two adjacent elements $\tau_k, \tau_\ell \in \meshT_N$ we have
\[
	\tilde{c}_G^{-1} \leq \frac{h_k}{h_\ell} \leq \tilde{c}_G.
\]
\end{assumption}
By this assumption we have for $\bar{h}_{k\ell} = \frac{1}{2} (h_k + h_\ell)$ with $c_G \coloneqq \frac{1}{2} (\tilde{c}_G + 1)$:
\[
	c_G^{-1} \leq \frac{\bar{h}_{k\ell}}{h_\ell} \leq c_G, \quad c_G^{-1} \leq \frac{\bar{h}_{k\ell}}{h_\ell} \leq c_G.
\]
\begin{lemma}
\label{thm:inverse-ineq}
For any discrete function $v_h \in \Shp(\meshT_N)$ there holds for all interior facets $\Gamma_{k \ell} \in \intfI_N$ the inverse inequalities
\begin{IEEEeqnarray*}{rCl}
	\| v_h \|_{L^2(\Gamma_{k\ell})} &\leq& c_I |\Gamma_{k \ell}|^{\frac{1}{2}} |\tau_\ell|^{-\frac{1}{2}} \| v_h \|_{L^2(\tau_\ell)}, \\
	\| \nabla_x v_h \|_{[L^2(\Gamma_{k\ell})]^d} &\leq& c_I |\Gamma_{k \ell}|^{\frac{1}{2}} |\tau_\ell|^{-\frac{1}{2}} \| \nabla_x v_h \|_{[L^2(\tau_\ell)]^d}, \\
	\| v_h \|_{H^1(\Gamma_{k\ell})} &\leq& c_I \bar{h}_{k\ell}^{-1} \| v_h \|_{L^2(\Gamma_{k \ell})}.
\end{IEEEeqnarray*}
For all $\tau_\ell \in \meshT_N$ there also holds
\begin{IEEEeqnarray*}{rCl}
	\| v_h \|_{H^1(\tau_\ell)} &\leq& c_I h_\ell^{-1} \| v_h \|_{L^2(\tau_\ell)}. \\
	\| v_h \|_{H^1(\partial \tau_\ell \cap \partial \Omega)} &\leq& c_I h_{\ell}^{-1} \| v_h \|_{L^2(\partial \tau_\ell \cap \partial \Omega)}.
\end{IEEEeqnarray*}
Here, $c_I > 0$ is a generic constant depending on the mesh and the polynomial degree.
\end{lemma}
\begin{proof}
For the statements see \cite[Lemma 2.2.3]{Neumueller}. For the last statement, see the proof of \cite[(4.28), (4.29)]{XuZou}.
\end{proof}
In the light of \cref{thm:Necas-Babuska}, we first define some energy norms with respect to the bilinear form $a(\cdot, \cdot)$, compare \cite{Neumueller}.
For a function $f \in H^s(\mathcal{T}_N)$ with $s > \frac{3}{2}$ we define
\begin{IEEEeqnarray*}{rCl}
\| f \|_A^2 &\coloneqq& \sum_{\ell = 1}^N \| \nabla_x f \|_{[L^2(\tau_\ell)]^d}^2 + \sum_{\Gamma_{k \ell} \in \mathcal{I}_N} \frac{\sigma}{\bar{h}_{k \ell}} \left\| \ljump f \rjump_{\Gamma_{k \ell}, x} \right\|_{[L^2(\Gamma_{k \ell})]^d}^2, \\
\| f \|_{A, *}^2 &\coloneqq& \| f \|_A^2 + \sum_{\Gamma_{k \ell} \in \mathcal{I}_N} \bar{h}_{k \ell} \left\| \lavg \nabla_x f \ravg_{\Gamma_{k \ell}} \right\|_{[L^2(\Gamma_{k \ell})]^d}^2, \\
\| f \|_{A, R}^2 &\coloneqq& \alpha \| f \|_{L^2(\Sigma_R)}^2.
\end{IEEEeqnarray*}
In the same vein, we define for $b(\cdot, \cdot)$ two norms for functions $f \in H^s(\mathcal{T}_N)$, $s \geq 1$:
\begin{IEEEeqnarray*}{rCl}
\| f \|_B^2 &\coloneqq& \sum_{\ell = 1}^N h_\ell \| \partial_t f \|_{L^2(\tau_\ell)}^2 + \| f(0) \|_{L^2(\Omega)}^2 + \| f(T) \|_{L^2(\Omega)}^2  + \sum_{\Gamma_{k \ell} \in \mathcal{I}_N} \left\| \ljump f \rjump_{\Gamma_{k \ell}, t} \right\|_{L^2(\Gamma_{k \ell})}^2, \\
\| f \|_{B, *}^2 &\coloneqq& \sum_{\ell = 1}^N h_\ell^{-1} \| f \|_{L^2(\tau_\ell)}^2 + \| f(T) \|_{L^2(\Omega)}^2  + \sum_{\Gamma_{k \ell} \in \mathcal{I}_N} \left\| \lupw f \rupw_{\Gamma_{k \ell}} \right\|_{L^2(\Gamma_{k \ell})}^2.
\end{IEEEeqnarray*}
In the light of the first condition of \cref{thm:Necas-Babuska}, we aim to prove an upper bound for $|A(\cdot, \cdot)|$ in some suitable norm. As of such, we'll estimate the individual terms of the bilinear first.
\begin{lemma}
\label{thm:asip-bounded}
The bilinear form $a^\sip(\cdot, \cdot)$ is bounded, i.e.
\[
	|a^\sip(f, v_h)| \leq c_2^a \| f \|_{A, *} \| v_h \|_A,
\]
for all $f \in H^s(\mathcal{T}_N)$ with $s > \frac{3}{2}$ and for all $v_h \in \Shp(\mathcal{T}_N)$ with an $h$-independent constant $c_2^a > 0$.
\end{lemma}
\begin{proof}
See \cite[Lemma 2.2.7]{Neumueller} and its prerequisite \cite[Lemma 2.2.6]{Neumueller}.
Alternatively, this has been proven in \cite[Lemma 4.16]{DiPietroErn}.
\end{proof}
\begin{lemma}
\label{thm:aR-bounded}
The bilinear form $a^R(\cdot, \cdot)$ defined by \cref{eq:aR-definition} is bounded, i.e.
\[
	|a^R(f, v_h)| \leq \| f \|_{A, R} \| v_h \|_{A, R}.
\]
\end{lemma}
\begin{proof}
Using the Cauchy-Schwarz inequality immediately yields:
\begin{IEEEeqnarray*}{rCl}
a^R(f, v_h) &\leq& \alpha \| f \|_{L^2(\Sigma_R)} \| v_h \|_{L^2(\Sigma_R)} \\
&=& \| f \|_{A,R} \| v_h \|_{A,R}.
\end{IEEEeqnarray*}
This is the claimed statement.
\end{proof}
\begin{lemma}
\label{thm:b-bounded}
The bilinear form $b(\cdot, \cdot)$ is bounded, i.e.
\[
	|b(f, v_h)| \leq \| f \|_{B, *} \| v_h \|_B
\]
for all $f \in H^s(\mathcal{T}_N)$ with $s \geq 1$ and for all $v_h \in \Shp(\mathcal{T}_N)$.
\end{lemma}
\begin{proof}
See \cite[Lemma 2.2.8]{Neumueller} for this statement.
\end{proof}
Motivated by \cref{thm:asip-bounded,thm:aR-bounded,thm:b-bounded}, we define for $f \in H^s(\meshT_N)$, $s > \frac{3}{2}$ the following energy norms:
\begin{equation}
\label{eq:dg-DG-norm}
\begin{IEEEeqnarraybox}[][c]{rCl}
\lDG f \rDG^2 &\coloneqq& \| f \|_A^2 + \| f \|_{A,R}^2 + \| f \|_B^2 \\
\lDGs f \rDGs^2 &\coloneqq& \| f \|_{A,*}^2 + \| f \|_{A,R}^2 + \| f \|_{B,*}^2
\end{IEEEeqnarraybox}
\end{equation}
\begin{theorem}
\label{thm:A-bounded}
The bilinear form $A(\cdot, \cdot)$ is bounded, i.e.
\[
	|A(f, v_h)| \leq c_2^A \lDGs f \rDGs \lDG v_h \rDG
\]
for all $f \in H^s(\meshT_N)$ with $s > \frac{3}{2}$ and for all $v_h \in \Shp(\meshT_N)$ with an $h$-independent constant $c_2^A > 0$. 
\end{theorem}
\begin{proof}
Combining \cref{thm:asip-bounded,thm:aR-bounded,thm:b-bounded} yields:
\begin{IEEEeqnarray*}{rCl}
|A(f, v_h)| &=& |a^\sip(f, v_h) + a^R(f, v_h) + b(f, v_h)| \\
&\leq& c_2^a \| f \|_{A, *} \| v_h \|_A + \| f \|_{A, R} \| v_h \|_{A, R} + \| f \|_{B, *} \| v_h \|_B \\
&\leq& \max \{ 1, c_2^a \} \lDGs f \rDGs \lDG v_h \rDG.
\end{IEEEeqnarray*}
By defining $c_2^A \coloneqq \max \{ 1, c_2^a \}$, we obtain the statement.
\end{proof}
With the boundedness given, we aim to prove a stability condition for $A(\cdot, \cdot)$.
\begin{lemma}
\label{thm:asip-lower-bound}
For $\sigma \geq 4 c_K$, with $c_K \coloneqq c_I^2 c_G c_{R_2}$, it holds that
\[
	a^\sip(f_h, f_h) \geq \frac{1}{2} \| f_h \|_A^2 \quad \forall f_h \in \Shp(\meshT_N).
\]
\end{lemma}
\begin{proof}
\cite[Lemma 2.2.9]{Neumueller}.
\end{proof}
\begin{proposition}
\label{thm:aR-lower-bound}
The bilinear form $a^R(\cdot, \cdot)$ fulfills
\[
	a^R(f_h, f_h) = \alpha \| f_h \|_{L^2(\Sigma_R)}^2 \quad \text{for all } f_h \in \Shp(\meshT_N).
\]
\end{proposition}
\begin{proof}
This follows immediately from the definition \cref{eq:aR-definition}.
\end{proof}
\begin{lemma}
\label{thm:b-lower-bound-weak}
The bilinear form $b(\cdot, \cdot)$ is bounded from below with
\[
	b(f_h, f_h) \geq \frac{1}{2} \left[ \| f_h(0) \|_{L^2(\Omega)}^2 + \| f_h(T) \|_{L^2(\Omega)}^2 + \sum_{\Gamma_{k\ell} \in \intfI_N} \left\| \ljump f_h \rjump_{\Gamma_{k \ell}, t} \right\|^2 \right]
\]
for all $f_h \in \Shp(\meshT_N)$.
\end{lemma}
\begin{proof}
Refer to \cite[Lemma 2.2.11]{Neumueller}.
\end{proof}
Bringing \cref{thm:asip-lower-bound,thm:aR-lower-bound,thm:b-lower-bound-weak} together, we obtain
\begin{theorem}
\label{thm:A-ellip-weak}
For all $f_h \in \Shp(\meshT_N)$ it holds that
\begin{IEEEeqnarray*}{rCl}
	A(f_h, f_h) &\geq& \frac{1}{2} \lDGw f_h \rDGw^2,
\end{IEEEeqnarray*}
where the seminorm $\lDGw \cdot \rDGw$ is defined as
\begin{IEEEeqnarray*}{rCl}
	\lDGw f_h \rDGw^2 &\coloneqq& \| f_h \|_A^2 + \| f_h \|_{A,R}^2 + \| f_h(0) \|_{L^2(\Omega)}^2 + \| f_h(T) \|_{L^2(\Omega)}^2 + \sum_{\Gamma_{k\ell} \in \intfI_N} \left\| \ljump f_h \rjump_{\Gamma_{k \ell}, t} \right\|^2.
\end{IEEEeqnarray*}
\end{theorem}
Given that $\Shp(\meshT_N)$ has a finite dimension, we know that any linear operator is continuous and that all norms are equivalent. Furthermore, $\lDGw \cdot \rDGw$ is defined as the sum of norms, so we immediately know that it fulfills the triangle inequality and possesses absolute homogeneity.
Hence, we know that $\lDGw \cdot \rDGw$ is a seminorm, and if we can prove that it separates points, it would be a norm on $\Shp(\meshT_N)$. One can easily see that it will separate points if $\Gamma_D$ has positive measure. In this case, \cref{thm:Necas-Babuska} will guarantee existence and uniqueness of a solution.

On the other hand, it cannot be a norm if for example $\alpha = 0$ and $\Gamma_D$ has zero measure. Consider the function $r(x, t) = t^2 - t$, which could be accurately sampled on $S_h^2(\meshT_N)$ for any $\meshT_N$: Since it's constant for fixed $t$, $\| r \|_A = 0$. Moreover, it is zero on the boundaries and continuous, one can see $\lDGw r \rDGw = 0$, but $r \neq 0$.

However, it turns out that it's not necessary to apply \cref{thm:Necas-Babuska} to prove that $A(\cdot, \cdot)$ is an isomorphism. Due to the finite dimension and linearity, proving that $A(\cdot, \cdot)$ is injective implies $A(\cdot, \cdot)$ will be an isomorphism.
\begin{theorem}
\label{thm:A-injective}
Let $\sigma \geq 4 c_K$, then $A(\cdot, \cdot)$ injective, i.e.\ if for a given $f_h \in \Shp(\meshT_N)$ it holds that
\[
	A(f_h, v_h) = 0 \quad \text{for all } v_h \in \Shp(\meshT_N),
\]
then $f_h = 0$ holds.
\end{theorem}
\begin{proof}
Let $f_h \in \Shp(\meshT_N)$ be arbitrary.
Due to \cref{thm:A-ellip-weak} and by choosing $v_h = f_h$ we have
\[
	0 = A(f_h, f_h) \geq \frac{1}{2} \lDGw f_h \rDGw^2,
\]
i.e.\ $\lDGw f_h \rDGw = 0$. Considering the single terms $\lDGw \cdot \rDGw$ is composed of, we conclude from $\| f_h \|_A = 0$ that
\begin{itemize}
	\item $\nabla_x f_h |_{\tau_\ell} = 0$ for all $\tau_\ell \in \meshT_N$,
	\item $\ljump f_h \rjump_{\Gamma_{k\ell}, x} = 0$ for all $\Gamma_{k\ell} \in \intfI_N$.
\end{itemize}
From $\| f_h \|_{A, R} = 0$ we conclude
\begin{itemize}
	\item $f_h = 0$ on $\Sigma_R$.
\end{itemize}
Finally, the remaining terms of $\lDGw \cdot \rDGw$ give us
\begin{itemize}
	\item $f_h = 0$ on $\Omega \times \{ 0 \}$,
	\item $f_h = 0$ on $\Omega \times \{ T \}$,
	\item $\ljump f_h \rjump_{\Gamma_{k\ell}, t} = 0$ for all $\Gamma_{k\ell} \in \intfI_N$.
\end{itemize}
Because $\ljump f_h \rjump_{\Gamma_{k\ell}, x}$ and $\ljump f_h \rjump_{\Gamma_{k\ell}, t}$ for all $\Gamma_{k\ell} \in \intfI_N$, we know that $f_h$ is continuous.

Choosing $v_h = \frac{\partial f_h}{\partial t} \in S_h^{p-1}(\meshT_N) \subset \Shp(\meshT_N)$, we observe that
\[
	0 = A \left(f_h, \frac{\partial f_h}{\partial t}\right) = a^\sip \left(f_h,  \frac{\partial f_h}{\partial t}\right) + a^R \left(f_h,  \frac{\partial f_h}{\partial t}\right) + b\left(f_h,  \frac{\partial f_h}{\partial t}\right) = b\left(f_h,  \frac{\partial f_h}{\partial t}\right),
\]
because $a^\sip(f_h, v_h)$ vanishes due to $f_h$ being continuous and $\nabla_x f_h = 0$, and $a^R(f_h, v_h)$ vanishes since $f_h = 0$ on the boundary $\Sigma_R$ holds.

Using \cref{eq:b-up-down-equal}, we furthermore conclude
\[
	0 = b\left(f_h,  \frac{\partial f_h}{\partial t}\right) = \sum_{\ell=1}^N \left\| \frac{\partial f_h}{\partial t} \right\|_{L^2(\tau_\ell)}^2,
\]
because $f_h = 0$ on $\Omega \times \{ 0 \}$ and $\ljump f_h \rjump_{\Gamma_{k\ell}, t}$ for all $\Gamma_{k\ell} \in \intfI_N$ imply that the other terms in \cref{eq:b-up-down-equal} vanish.

Finally, we know that $f_h$ is a continuous function with zero derivatives on $\meshT_N$ and zero boundary values, i.e.\ $f_h = 0$ has to hold.  
\end{proof}
Summarizing \cref{thm:A-injective} and the previously made remarks, we obtain
\begin{theorem}
\label{thm:A-bijective}
Let \cref{as:mesh-assumptions} hold and be $\sigma \geq 4 c_K$. There exists a unique solution $f_h \in \Shp(\meshT_N)$ of \cref{eq:dg-discrete-form}.
\end{theorem}

While \cref{thm:b-lower-bound-weak} was sufficient to prove existence and uniqueness of a solution, it will not suffice to prove a convergence result. With an $\inf$-$\sup$-condition (or an ellipticity estimate, which implies it) and boundedness, a convergence result could usually be shown, like \cite[3.7 Hilfssatz]{Braess}.

For such a result we will have to make a slightly stronger assumption on the mesh $\meshT_N$. In that case, we can prove via a special test function that we have an $\inf$-$\sup$-condition given and can prove convergence.

Remembering the equivalent formulation of $b(\cdot, \cdot)$ given as \cref{eq:b-up-down-equal}, we observe that the first term is an integral between the time derivative of the function $f_h$ and $v_h$ itself.
As of such, an approach is made where our special test function is element-wise defined as the product of the time derivative and a special, so called mesh function, which acts as a linear, piecewise weight scale, see for example \cite{Neumueller}.
\begin{definition}
Let $\bar{h}$ be a piecewise linear function on $\meshT_N$, i.e.\ $\bar{h}|_{\tau_\ell} \in \polyP_1(\tau_\ell)$ for all $\tau_\ell \in \meshT_N$, with the property
\[
	c_g^{-1} h_\ell \leq \bar{h}(x, t) \leq c_g h_\ell \quad \text{for all } (x, t) \in \bar{\tau}_\ell
\]
and $c_g \geq 1$. Then $\bar{h}$ is called a mesh function.
\end{definition}
With this definition, we can follow \cite{Neumueller} and introduce for a given an $f_h \in \Shp(\meshT_N)$ and a given mesh function $\bar{h}$, the function $w_h \in \Shp(\meshT_N)$ as
\begin{equation}
\label{eq:wh}
	w_h|_{\tau_\ell} \coloneqq \bar{h}|_{\tau_\ell} \partial_t f_h|_{\tau_\ell} \in \polyP^p(\tau_\ell)
\end{equation}
for all $\tau_\ell \in \meshT_N$. Obviously, $w_h \in \Shp(\meshT_N)$ holds by definition.

Our approach will be to choose for a given mesh function $\bar{h}$ and a given $f_h \in \Shp(\meshT_N)$ the test function
\[
	v_h = f_h + \delta w_h,
\]
where $\delta > 0$ with the motivation of taking the limit $\delta \to 0$.

More specifically, we'll use the following mesh function, which is constant on $\meshT_N$:
\begin{equation}
\label{eq:mesh-func}
	\bar{h} = \frac{1}{N} \sum_{\ell = 1}^N h_\ell.
\end{equation}

\begin{definition}
If there is a $c_q \geq 1$ and a $h > 0$ such that
\[
	c_q^{-1} h \leq h_\ell \leq c_q h \quad \text{for all } \tau_\ell \in \meshT_N,
\]
we call $\meshT_N$ quasi-uniform.
\end{definition}
Assuming that $\meshT_N$ does not only fulfill \cref{as:mesh-assumptions}, but is also quasi-uniform, i.e.\ one can see that \cref{eq:mesh-func} is indeed a mesh function:
\[
	\bar{h} = \frac{1}{N} \sum_{k = 1}^N h_k \leq \frac{1}{N} \sum_{k=1}^N c_q h \leq c_q^2 \frac{1}{N} \sum_{k=1}^N c_q^{-1} h \leq c_q^2 \frac{1}{N} \sum_{k=1}^N h_\ell = c_q^2 h_\ell.
\]
Analogously, one obtains $c_q^{-2} h_\ell \leq \bar{h}$. Hence, $\bar{h}$ is a mesh function for $c_g = c_q^2$.
Due to this, we make the following assumption:
\begin{assumption}
From here on, we assume that $\meshT_N$ is always quasi-uniform.
\end{assumption}
With this mesh function, one can procure stability results, which we introduce in the following.
Here, we cite \cite{Neumueller}, who followed ideas presented in \cite{EggerSchoberl} in order to prove these results.
\begin{lemma}
\label{thm:wh-b-bound}
For $f_h \in \Shp(\meshT_N)$ let $w_h \in \Shp(\meshT_N)$ be defined by \cref{eq:mesh-func}.
Then for $\delta > 0$ there exists a $c_1^b(\delta) \coloneqq \frac{1}{2} \min \left\{ 1, \delta c_g^{-1}, 1 - 2 c_I^2 c_{R_2} c_g^3 \delta \right\} > 0$ independent of $f_h$ such that
\[
	b(f_h, f_h + \delta w_h) \geq c_1^b(\delta) \| f_h \|_B^2.
\]
Moreover, there exist a $c_I^b > 0$ such that
\[
	\| w_h \|_B \leq c_I^b \| f_h \|_B.
\]
\end{lemma}
\begin{proof}
Refer to \cite[Lemma 2.2.14]{Neumueller} and \cite[Lemma 2.2.15]{Neumueller}.
\end{proof}
\begin{lemma}
\label{thm:wh-A-bound}
For $f_h \in \Shp(\meshT_N)$ let $w_h \in \Shp(\meshT_N)$ be defined by \cref{eq:mesh-func}. Then there exists a constant $c_I^a > 0$ such that
\[
	\| w_h \|_A \leq c_I^a \| f_h \|_A.
\]
\end{lemma}
\begin{proof}
See \cite[Lemma 2.2.19]{Neumueller}.
\end{proof}
\begin{lemma}
\label{thm:wh-aR-bound}
For $f_h \in \Shp(\meshT_N)$ let $w_h \in \Shp(\meshT_N)$ be defined by \cref{eq:mesh-func}. Then there exists a constant $c_I^{a, R} > 0$ such that
\[
	\| w_h \|_{A, R} \leq c_I^{a, R} \| f_h \|_{A, R}.
\]
\end{lemma}
\begin{proof}
We first show $\| w_h \|_{L^2(\Sigma_R)}^2 \leq c \| f_h \|_{L^2(\Sigma_R)}^2$:
\begin{IEEEeqnarray*}{rCl}
	\| w_h \|_{L^2(\Sigma_R)}^2 &=& \sum_{\substack{\tau_\ell \in \meshT_N \\ \partial \tau_\ell \cap \Sigma_R \neq \emptyset}} \| w_h \|_{L^2(\partial \tau_\ell \cap \Sigma_R)}^2 \\
	&=& \sum_{\substack{\tau_\ell \in \meshT_N \\ \partial \tau_\ell \cap \Sigma_R \neq \emptyset}} \| \bar{h} \partial_t f_h \|_{L^2(\partial \tau_\ell \cap \Sigma_R)}^2. \\
\noalign{\noindent By using that $\bar{h}$ as defined in \cref{eq:mesh-func} is constant:\vspace{\jot}}
	\| w_h \|_{L^2(\Sigma_R)}^2  &=& \sum_{\substack{\tau_\ell \in \meshT_N \\ \partial \tau_\ell \cap \Sigma_R \neq \emptyset}} \bar{h}^2 \| \partial_t f_h \|_{L^2(\partial \tau_\ell \cap \Sigma_R)}^2. \\
\noalign{\noindent Since $\bar{h}$ was a mesh function, we can estimate this as follows:\vspace{\jot}}
	\| w_h \|_{L^2(\Sigma_R)}^2 &\leq& c_g^2 \sum_{\substack{\tau_\ell \in \meshT_N \\ \partial \tau_\ell \cap \Sigma_R \neq \emptyset}} h_\ell^2 \| \partial_t f_h \|_{L^2(\partial \tau_\ell \cap \Sigma_R)}^2. \\
\noalign{\noindent The inverse inequality of \cref{thm:inverse-ineq} transforms this into:\vspace{\jot}}
	\| w_h \|_{L^2(\Sigma_R)}^2 &\leq& c_g^2 c_I^2 \sum_{\substack{\tau_\ell \in \meshT_N \\ \partial \tau_\ell \cap \Sigma_R \neq \emptyset}} h_\ell^2 h_\ell^{-2} \| f_h \|_{L^2(\partial \tau_\ell \cap \Sigma_R)}^2 \\
	&=& c_g^2 c_I^2 \| f_h \|_{L^2(\Sigma_R)}^2.
\end{IEEEeqnarray*}
Therefore we have
\[
	\| w_h \|_{A,R} = \alpha \| w_h \|_{L^2(\Sigma_R)} \leq \alpha c_g^2 c_I^2 \| f_h \|_{L^2(\Sigma_R)} = c_g^2 c_I^2 \| f_h \|_{A, R}
\]
and as of such have proven the statement for a choice of $c_I^{a, R} = c_g c_I$.
\end{proof}
As a next step, we will combine all previous estimates to obtain stability and boundedness estimates for the bilinear form $A(\cdot, \cdot)$ itself.
However, we first need the following result for the proof:
\begin{lemma}
\label{thm:Astar-to-A-estimate}
For a function $f_h \in \Shp(\meshT_N)$, the norm $\| f_h \|_{A, *}$ can be estimated as follows:
\[
	\| f_h \|_{A, *} \leq \sqrt{1 + c_K} \| f_h \|_A,
\]
with the constant $c_K$ as introduced before.
\end{lemma}
\begin{proof}
In \cite[Lemma 2.2.6]{Neumueller} it is being shown that
\[
\sum_{\Gamma_{k \ell} \in \mathcal{I}_N} \bar{h}_{k \ell} \left\| \lavg \nabla_x f_h \ravg_{\Gamma_{k \ell}} \right\|_{[L^2(\Gamma_{k \ell})]^d}^2 \leq c_K \sum_{\ell=1}^N \left\| \nabla_x f_h \right\|_{[L^2(\tau_\ell)]^d}^2,
\]
for all $f_h \in \Shp(\meshT_N)$ and $c_K = c_K(c_I, c_G, c_{R_2})$.

Therefore
\begin{IEEEeqnarray*}{rCl}
	\| f_h \|_{A, *}^2 &=& \| f \|_A^2 + \sum_{\Gamma_{k \ell} \in \mathcal{I}_N} \bar{h}_{k \ell} \left\| \lavg \nabla_x f \ravg_{\Gamma_{k \ell}} \right\|_{[L^2(\Gamma_{k \ell})]^d}^2 \\
	&\leq& \| f_h \|_A^2 + c_K \sum_{\ell=1}^N \left\| \nabla_x f_h \right\|_{[L^2(\tau_\ell)]^d}^2 \\
	&\leq& \| f_h \|_A^2 + c_K \| f_h \|_A^2.
\end{IEEEeqnarray*}
Hence, we conclude the statement.
\end{proof}
\begin{theorem}
\label{thm:Astab-est}
Let $\meshT_N$ be a quasi-uniform decomposition and let $\sigma \geq 4 c_K$ then the following stability estimate holds:
\[
	\sup_{0 \neq v_h \in \Shp(\meshT_N)} \frac{A(f_h, v_h)}{\lDG v_h \rDG} \geq c_S^A \lDG f_h \rDG \quad \text{for all } f_h \in \Shp(\meshT_N).
\]
\end{theorem}
\begin{proof}
Let a function $f_h \in \Shp(\meshT_N)$ be given.
The case $f_h = 0$ is trivial, so we assume $f_h \neq 0$ in the following.
We use the specific test function $v_h = f_h + \delta w_h$, where $\delta > 0$ and $w_h$ is stemming from the mesh function $\bar{h}$ defined in \cref{eq:mesh-func}.
Because $f_h \neq 0$, we can assume that $\delta$ is small enough so that $v_h \neq 0$. Furthermore $v_h \in \Shp(\meshT_N)$ holds and we have
\begin{IEEEeqnarray*}{rCl}
	\sup_{0 \neq v_h \in \Shp(\meshT_N)} \frac{A(f_h, v_h)}{\lDG v_h \rDG} &\geq& \frac{A(f_h, f_h + \delta w_h)}{\lDG f_h + \delta w_h \rDG} \\
	&=& \frac{a^\sip(f_h, f_h + \delta w_h) + a^R(f_h, f_h + \delta w_h) + b(f_h, f_h + \delta w_h)}{\lDG f_h + \delta w_h \rDG}
\end{IEEEeqnarray*}
By applying estimates from the previous \cref{thm:asip-lower-bound,thm:asip-bounded}, we obtain
\begin{IEEEeqnarray*}{rCl}
	a^\sip(f_h, f_h + \delta w_h) &=& a^\sip(f_h, f_h) + \delta a^\sip(f_h, w_h) \\
	&\geq& \frac{1}{2} \| f_h \|_A^2 - c_2^a \delta \| f_h \|_{A,*} \| w_h \|_A
\end{IEEEeqnarray*}
Using \cref{thm:wh-A-bound}, one has $\| w_h \|_A \leq c_I^a \| f_h \|_A$.

Overall we obtain for this term
\begin{equation}
\label{eq:Astab-asip-est}
\begin{IEEEeqnarraybox}[][c]{rCl}
	a^\sip(f_h, f_h + \delta w_h) &\geq& \frac{1}{2} \| f_h \|_A^2 - c_2^a \delta \| f_h \|_{A,*} \| w_h \|_A \\
	&\overset{\text{\cref{thm:Astar-to-A-estimate}}}{\geq}& \frac{1}{2} \| f_h \|_A^2 - c_2^a \sqrt{1 + c_K} c_I^a \delta \| f_h \|_{A}^2 \\
	&=& \left( \frac{1}{2} - c_2^a \sqrt{1 + c_K} c_I^a \delta \right) \| f_h \|_{A}^2
\end{IEEEeqnarraybox}
\end{equation}
For the second term we can estimate:
\begin{IEEEeqnarray*}{rCl}
	a^R(f_h, f_h + \delta w_h) &=& a^R(f_h, f_h) + \delta a^R(f_h, w_h) \\
	&\overset{\text{\cref{thm:aR-bounded,thm:aR-lower-bound}}}{\geq}& \| f_h \|_{A, R}^2 - \delta \| f_h \|_{A, R} \| w_h \|_{A, R} \\
	&\overset{\text{\cref{thm:wh-aR-bound}}}{\geq}& \left( 1 - \delta c_I^{a, R} \right) \| f_h \|_{A, R}^2
\end{IEEEeqnarray*}
Finally, in order to treat the last term, we can use \cref{thm:wh-b-bound}. Overall we obtain
\begin{IEEEeqnarray*}{rCl}
	A(f_h, v_h) &\geq& \left( \frac{1}{2} - c_2^a \sqrt{1 + c_K} c_I^a \delta \right) \| f_h \|_{A}^2 + \left( 1 - \delta c_I^{a, R} \right) \| f_h \|_{A, R}^2 + c_1^b(\delta) \| f_h \|_B^2 \\
	&=& \frac{1}{2} \Bigg[ \left( 1 - 2 c_2^a \sqrt{1 + c_K} c_I^a \delta \right) \| f_h \|_{A}^2 + \left( 2 - 2\delta c_I^{a, R} \right) \| f_h \|_{A, R}^2 \\
	&& \qquad {} + \min \left\{ 1, \delta c_g^{-1}, 1 - 2 c_I^2 c_{R_2} c_g^3 \delta \right\} \| f_h \|_B^2 \Bigg] \\
	&\geq& \min \left\{ 1 - 2 c_2^a \sqrt{1 + c_K} c_I^a \delta, 2 - 2\delta c_I^{a, R}, 1, \delta c_g^{-1}, 1 - 2 c_I^2 c_{R_2} c_g^3 \delta \right\} \lDG f_h \rDG^2
\end{IEEEeqnarray*}
By choosing $\delta = \delta^*$ with
\[
	\delta^* \coloneqq \min \left\{ \frac{1}{c_g^{-1} + 2 c_I^2 c_{R_2} c_g^3}, \frac{1}{c_g^{-1} +  c_I^{a,R}}, \frac{1}{c_g^{-1} + 2 c_2^a \sqrt{1 + c_K} c_I^a} \right\}
\]
we obtain
\[
	A(f_h, v_h) \geq \min \{ 1, \delta^* c_g^{-1} \} \lDG f_h \rDG^2
\]

Additionally, we also have to estimate the denominator. For this we employ \cref{thm:wh-A-bound,thm:wh-b-bound,thm:wh-aR-bound}:
\begin{IEEEeqnarray*}{rCl}
	\lDG f_h + \delta w_h \rDG &=& \| f_h + \delta w_h \|_A + \| f_h + \delta w_h \|_{A, R} + \| f_h + \delta w_h \|_B \\
	&\leq& \lDG f_h \rDG + \delta ( \| w_h \|_A + \| w_h \|_{A, R} + \| w_h \|_B ) \\
	&\leq& \lDG f_h \rDG + \delta ( c_I^a \|f_h \|_A + c_I^{a, R} \| f_h \|_{A, R} + c_I^b \| f_h \|_B ) \\
	&\leq& \lDG f_h \rDG + \delta \max\{c_I^a, c_I^{a, R}, c_I^b\} \lDG f_h \rDG \\
	&=& (1 + \delta \max\{c_I^a, c_I^{a, R}, c_I^b\} ) \lDG f_h \rDG
\end{IEEEeqnarray*}
Combining the two inequalities:
\begin{IEEEeqnarray*}{rCl}
	\sup_{0 \neq v_h \in \Shp(\meshT_N)} \frac{A(f_h, v_h)}{\lDG v_h \rDG} &\geq& \frac{A(f_h, f_h + \delta w_h)}{\lDG f_h + \delta w_h \rDG} \\
	&\geq& \frac{ \min\{ 1, \delta^* c_g^{-1} \} }{2 ( 1 + \delta^* \max\{c_I^a, c_I^{a, R}, c_I^b\} )} \lDG f_h \rDG \\
	&=& c_S^A \lDG f_h \rDG.
\end{IEEEeqnarray*}
With this, we have proven the stability estimate we were aiming for.
\end{proof}
\begin{theorem}
\label{thm:A-convergence-abstract}
Let $\meshT_N$ be a quasi-uniform decomposition and let $f \in H^s(\meshT_N)$, $s > \frac{3}{2}$, be the exact solution of \cref{eq:dg-model-prob}.
For $\sigma \geq 4 c_K$ let $f_h \in \Shp(\meshT_N)$ be the solution of the discrete variational problem \cref{eq:dg-discrete-prob}.
Then the following error estimate holds:
\[
	\lDG f - f_h \rDG \leq \inf_{z_h \in \Shp(\meshT_N)} \left[ \lDG f - z_h \rDG + \frac{c_2^A}{c_S^A} \lDGs f - z_h \rDGs \right].
\]
\end{theorem}
\begin{proof}
From \cref{thm:Astab-est}, we conclude for any discrete function $z_h \in \Shp(\meshT_N)$:
\begin{IEEEeqnarray*}{rCl}
	c_S^A \lDG z_h - f_h \rDG &\leq& \sup_{0 \neq v_h \in \Shp(\meshT_N)} \frac{A(z_h - f_h, v_h)}{\lDG v_h \rDG}. \\
\noalign{\noindent Using \cref{eq:dg-Galerkin-orthogonality}, we can insert the exact solution: \vspace{\jot}}
	c_S^A \lDG z_h - f_h \rDG &=& \sup_{0 \neq v_h \in \Shp(\meshT_N)} \frac{A(f - f_h - (f - z_h), v_h)}{\lDG v_h \rDG} \\
	&\overset{\text{\cref{eq:dg-Galerkin-orthogonality}}}{=}& \sup_{0 \neq v_h \in \Shp(\meshT_N)} \frac{A(z_h - f, v_h)}{\lDG v_h \rDG}. \\
\noalign{\noindent Using the boundedness estimate \cref{thm:A-bounded}: \vspace{\jot}}
	c_S^A \lDG z_h - f_h \rDG &\leq& \sup_{0 \neq v_h \in \Shp(\meshT_N)} \frac{c_2^A \lDGs z_h - f \rDGs \lDG v_h \rDG}{\lDG v_h \rDG} \\
	&=& \sup_{0 \neq v_h \in \Shp(\meshT_N)} c_2^A \lDGs z_h - f \rDGs \\
	&=& c_2^A \lDGs f - z_h \rDGs.
\end{IEEEeqnarray*}
Using the triangle inequality we obtain:
\begin{IEEEeqnarray*}{rCl}
	\lDG f - f_h \rDG &\leq& \lDG f - z_h \lDG + \lDG z_h - f_h \rDG \\
	&\leq& \lDG f - z_h \lDG + \frac{c_2^A}{c_S^A} \lDGs f - z_h \rDGs.
\end{IEEEeqnarray*}
This is exactly the statement we aimed to prove.
\end{proof}
As a next step, we need to be able to estimate the approximation terms in \cref{thm:A-convergence-abstract}. For this purpose, we introduce the $L^2$-projection and estimate the error in the $\lDG \cdot \rDG$ and $\lDGs \cdot \rDGs$ norms.
\begin{lemma}
\label{thm:L2-approximation-props}
Let $\meshT_N$ be a decomposition of the space-time domain $Q$.
For an element $\tau_\ell \in \meshT_N$ let $f \in H^s(\tau_\ell)$, $s \geq 0$, be a given function.
By $Q_\ell f \in \polyP^p(\tau_\ell)$ we denote the local $L^2$-projection on $\tau_\ell$ as
\[
	\langle Q_\ell f, v_h \rangle_{L^2(\tau_\ell)} = \langle f, v_h \rangle_{L^2(\tau_\ell)} \quad \text{for all } v_h \in \polyP^p(\tau_\ell).
\]
Then the following error estimates for the local $L^2$-projection hold:
For $s \in \N_0$ and $0 \leq \mu \leq s$ there holds
\[
	| f - Q_\ell f |_{H^\mu(\tau_\ell)} \leq ch_\ell^{\min\{s, p+1\} - \mu} |f|_{H^s(\tau_\ell)}.
\]
For $s \in \N$ there holds the $L^2$-error estimate on the boundary
\[
	\| f - Q_\ell f \|_{L^2(\partial \tau_\ell)} \leq ch_\ell^{\min\{s, p+1\} - \frac{1}{2}} |f|_{H^s(\tau_\ell)}.
\]
For $s \in \N$ with $s \geq 2$ the $L^2$-error on the boundary $\partial \tau_\ell$ for the gradient can be estimated by
\[
	\| \nabla_x (f - Q_\ell f) \|_{[L^2(\partial \tau_\ell)]^d} \leq ch_\ell^{\min\{s, p+1\} - \frac{3}{2}} |f|_{H^s(\tau_\ell)}.
\]
\end{lemma}
\begin{proof}
The Lemma has been proven in \cite[Lemma 1.58 and Lemma 1.59]{DiPietroErn}.
\end{proof}
We define the global $L^2$-projection $Q_{\meshT_N} f \in \Shp(\meshT_N)$ such that
\[
	Q_{\meshT_N} f|_{\tau_\ell} \coloneqq Q_\ell f \quad \text{for all } \tau_\ell \in \meshT_N.
\]
\begin{lemma}
\label{thm:L2proj-A-error}
For $f \in H^s(\meshT_N)$ with $s \geq 1$ the following error estimate in the energy norm $\| \cdot \|_A$ holds
\[
	\| f - Q_{\meshT_N} f \|_A \leq c \left[ \sum_{\ell=1}^N h_\ell^{2 \min \{ s, p+ 1 \} -2 } |f|^2_{H^s(\tau_\ell)} \right]^\frac{1}{2}.
\]
\end{lemma}
\begin{proof}
See \cite[Lemma 2.2.24]{Neumueller}.
\end{proof}
\begin{lemma}
\label{thm:L2proj-AR-error}
For $f \in H^s(\meshT_N)$ with $s \geq 1$ the following error estimate in the energy norm $\| \cdot \|_{A,R}$ holds
\[
	\| f - Q_{\meshT_N} f \|_{A,R} \leq \sqrt{\alpha} c \left[ \sum_{\ell=1}^N h_\ell^{2 \min \{ s, p+ 1 \} - 1} |f|^2_{H^s(\tau_\ell)} \right]^\frac{1}{2}.
\]
\end{lemma}
\begin{proof}
The case $\alpha = 0$ is trivial, as the norm will vanish. Therefore, let $\alpha \neq 0$.

Let $f \in H^s(\meshT_N)$ with $s \geq 1$.
\begin{IEEEeqnarray*}{rCl}
	\| f - Q_{\meshT_N} f \|_{L^2(\Sigma_R)}^2 &=& \sum_{\substack{\tau_\ell \in \meshT_N \\ \partial \tau_\ell \cap \Sigma_R}} \| f - Q_{\meshT_N} f \|_{L^2(\partial \tau_\ell \cap \Sigma_R)} \\
	&\leq& \sum_{\ell=1}^N \| f - Q_{\meshT_N} f \|_{L^2(\partial \tau_\ell)}^2 \\
\noalign{\noindent Applying \cref{thm:L2-approximation-props} we estimate this as}
	&\leq& c^2 \sum_{\ell=1}^N h_\ell^{2 \min \{ s, p+ 1 \} - 1} |f|^2_{H^s(\tau_\ell)}.
\end{IEEEeqnarray*}
By definition of $\| \cdot \|_{A,R}$:
\[
	\| f - Q_{\meshT_N} f \|_{A,R}^2 \leq \alpha c^2 \sum_{\ell=1}^N h_\ell^{2 \min \{ s, p+ 1 \} - 1} |f|^2_{H^s(\tau_\ell)}.
\]
\end{proof}
\begin{lemma}
\label{thm:L2proj-Astar-error}
For $f \in H^s(\meshT_N)$ with $s \geq 2$ the following error estimate in the energy norm $\| \cdot \|_{A,*}$ holds
\[
	\| f - Q_{\meshT_N} f \|_{A,*} \leq c \left[ \sum_{\ell=1}^N h_\ell^{2 \min \{ s, p+ 1 \} -2 } |f|^2_{H^s(\tau_\ell)} \right]^\frac{1}{2}.
\]
\end{lemma}
\begin{proof}
See \cite[Lemma 2.2.25]{Neumueller}.
\end{proof}
\begin{lemma}
\label{thm:L2proj-B-error}
For $f \in H^s(\meshT_N)$ with $s \geq 1$ the following error estimate in the energy norms $\| \cdot \|_{B}$ and $\| \cdot \|_{B,*}$ holds
\begin{IEEEeqnarray*}{rCl}
	\| f - Q_{\meshT_N} f \|_{B} &\leq& c \left[ \sum_{\ell=1}^N h_\ell^{2 \min \{ s, p+ 1 \} - 1 } |f|^2_{H^s(\tau_\ell)} \right]^\frac{1}{2}, \\
	\| f - Q_{\meshT_N} f \|_{B,*} &\leq& c \left[ \sum_{\ell=1}^N h_\ell^{2 \min \{ s, p+ 1 \} - 1 } |f|^2_{H^s(\tau_\ell)} \right]^\frac{1}{2}.
\end{IEEEeqnarray*}
\end{lemma}
\begin{proof}
See \cite[Lemma 2.2.27]{Neumueller}.
\end{proof}
\begin{theorem}
\label{thm:dg-convergence}
Let $\meshT_N$ be a quasi-uniform decomposition and let $f \in H^s(\meshT_N)$, $s \geq 2$, be the exact solution of \cref{eq:dg-model-prob} and for $\sigma \geq 4 c_K$ let $f_h \in \Shp(\meshT_N)$ be the solution of the discrete variational problem \cref{eq:dg-discrete-form}.
Then the following error estimate holds
\[
	\lDG f - f_h \rDG \leq c \max\{ 1, \alpha \} h^{\min \{ s, p+1\} -1} |f|_{H^s(\meshT_N)}.
\]
\end{theorem}
\begin{proof}
We apply \cref{thm:A-convergence-abstract} and use the $L^2$-projection $Q_{\meshT_N} u \in \Shp(\meshT_N)$ to obtain:
\begin{IEEEeqnarray*}{rCl}
	\lDG f - f_h \rDG &\leq& \inf_{z_h \in \Shp(\meshT_N)} \left[ \lDG f - z_h \rDG + \frac{c_2^A}{c_S^A} \lDGs f - z_h \rDGs \right] \\
	&\leq& \lDG f - Q_{\meshT_N} f \rDG + \frac{c_2^A}{c_S^A} \lDGs f - Q_{\meshT_N} f \rDGs. \\
\noalign{\noindent Using the previously derived results \cref{thm:L2proj-A-error,thm:L2proj-AR-error,thm:L2proj-Astar-error,thm:L2proj-B-error}, we obtain:}
	\lDG f - f_h \rDG &=& \left[ \| f - Q_{\meshT_N} f \|_A^2 + \| f - Q_{\meshT_N} f \|_{A, R}^2 + \| f - Q_{\meshT_N} f \|_B^2 \right]^\frac{1}{2} \\
	&& \quad {} + \frac{c_2^A}{c_S^A} \left[ \| f - Q_{\meshT_N} f \|_{A,*}^2 + \| f - Q_{\meshT_N} f \|_{A, R}^2 + \| f - Q_{\meshT_N} f \|_{B,*}^2 \right]^\frac{1}{2} \\
	&\leq& c \max\{ 1, \alpha \} \left(1 + \frac{c_2^A}{c_S^A} \right)\left[ \sum_{\ell=1}^N (1 + h_\ell) h_\ell^{2 \min \{ s, p+ 1 \} - 2 } |f|^2_{H^s(\tau_\ell)} \right]^\frac{1}{2} \\
	&\leq& c \max\{ 1, \alpha \} \left(1 + \frac{c_2^A}{c_S^A} \right)\left[ \sum_{\ell=1}^N h_\ell^{2 \min \{ s, p+ 1 \} - 2 } |f|^2_{H^s(\tau_\ell)} \right]^\frac{1}{2}. \\
\noalign{\noindent Exploiting the quasi-uniform property of $\meshT_N$, we obtain}
	\lDG f - f_h \rDG &\leq& c c_g \max\{ 1, \alpha \} \left(1 + \frac{c_2^A}{c_S^A} \right) h^{\min \{ s, p+1 \} - 1 } |f|_{H^s(\meshT_N)}.
\end{IEEEeqnarray*}
This proves the convergence result we were looking for.
\end{proof}
\begin{remark}
Unlike all the other estimates needed in the proof of \cref{thm:dg-convergence}, the \cref{thm:L2proj-Astar-error} needs $s \geq 2$ and therefore increases the requirements of the theorem to $s \geq 2$. The reason why $s \geq 2$ is needed in order to prove \cref{thm:L2proj-Astar-error} is that the last estimate of \cref{thm:L2-approximation-props} is needed for estimating the term
\[
	\sum_{\Gamma_{k \ell} \in \mathcal{I}_N} \bar{h}_{k \ell} \left\| \lavg \nabla_x (f - Q_{\meshT_N} f) \ravg_{\Gamma_{k \ell}} \right\|_{[L^2(\Gamma_{k \ell})]^d}^2
\]
in the norm $\| f - Q_{\meshT_N} f \|_{A,*}^2$.

One might think that it would suffice to have $H^2$ ``only in space direction'' in order for the estimate of \cref{thm:L2-approximation-props} to hold, i.e.\ in the terms of \cref{def:Bochner-space}, that $f \in L^2(0, T; H^2(\Omega)) \cap H^1(0, T; L^2(\Omega))$ holds.
However, this does not seem to be the case.
In order to prove the estimate of the gradient on the interfaces, one needs the continuous trace inequality on the elements, see \cite[Lemma 1.49]{DiPietroErn} and subsequently applies the estimate for $|f - Q_\ell f|_{H^\mu(\tau_\ell)}$ for $\mu \geq 2$.
Overall, this approach does not seem to be portable to a mixed regularity setup, as the trace inequality used requires an estimation of the term $\| \nabla ( \nabla_x (f - Q_\ell f) ) \|_{[L^2(\tau_\ell)]^d}$.
Even if one could estimate this by $\| \nabla_x^2 (f - Q_\ell f) ) \|_{[L^2(\tau_\ell)]^d}$ using different techniques, estimating that term would likely prove difficult as well, because the proof of \cref{thm:L2-approximation-props} is classically performed by estimating them against polynomial approximations that require $f \in H^2(\meshT_N)$ still.

In some sense, this poses a problem: There exists a variety of results for situations where an improved regularity of $f \in L^2(0, T; H^2(\Omega)) \cap H^1(0, T; L^2(\Omega))$ holds for the exact solution, for example if $\Omega$ is polynomial and convex, compare for example \cite[7.1.3 Regularity]{Evans}.
However, a result of $f \in H^2(0, T; L^2(\Omega))$ is difficult to obtain under a generic setup.

The regularity assumption $f \in H^s(\meshT_N)$, $s \geq 2$ can be relaxed a bit nonetheless: It suffices to ask for $f \in W_p^2(\meshT_N)$ with $p \in ( \frac{2d}{d+2}, 2 ]$, see \cite[4.2.5 Analysis for Low-Regularity Solutions]{DiPietroErn}, as originally presented in \cite{Wihler} for two space dimensions and in \cite{DiPietroErn-Low} for any space dimension.
\end{remark}
\end{document}