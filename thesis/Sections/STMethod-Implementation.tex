\documentclass[../thesis.tex]{subfiles}

\begin{document}
\chapter{Numerical treatment of the control problem}
\label{sec:st-numerics}
In \cref{sec:Disc-Control-Problem} we derived optimality conditions and convergence for an approximation to the optimal control heating problem for both, boundary and interior heat sources.
We now want to discuss numerical methods for solving the formulations established.
For now, we consider the case with no restrictions made to the control, i.e.\ $u_a = -\infty$ and $u_b = \infty$. In that setup, the projection formula becomes a simple equation: $u_h = - \lambda^{-1} \beta p_h$.
A key advantage of the discontinuous Galerkin approach presented here is that we can immediately assemble a linear equation system for this case and solve the already discretized system with a common equation solver.
\section{Establishing a linear optimality system in the case with no control restrictions}
\label{sec:num-unconstrained}
In this section, we assume $u_a = -\infty$ and $u_b = \infty$, as well as $\beta$ being a constant with $\beta > 0$. 
For the boundary heat system we obtain under this setup an optimality system of
\begin{IEEEeqnarray*}{c}
\begin{IEEEeqnarraybox}{rCl"rCl}
A(y_h, v_h) &=& \langle \beta u_h, v_h \rangle_{L^2(\Sigma_R)} + \langle y_0, v_h(0) \rangle_{L^2(\Omega)} & A(v_h, p_h) &=& \langle y_h(T) - y_\Omega, v_h(T) \rangle_{L^2(\Omega)}
\end{IEEEeqnarraybox} \\
u_h = - \lambda^{-1} \beta p_h.
\end{IEEEeqnarray*}
holding for all $v_h \in \Shp(\meshT_N)$.
By substituting $u_h$ with the representation given by the variational equality, we obtain the following system:
\begin{IEEEeqnarray*}{rCl"l}
A(y_h, v_h) &=& - \langle \beta^2 \lambda^{-1} p_h, v_h \rangle_{L^2(\Sigma_R)} + \langle y_0, v_h(0) \rangle_{L^2(\Omega)}& \forall v_h \in \Shp(\meshT_N) \\
A(v_h, p_h) &=& \langle y_h(T) - y_\Omega, v_h(T) \rangle_{L^2(\Omega)} & \forall v_h \in \Shp(\meshT_N)
\end{IEEEeqnarray*}
If we bring the terms depending on $y_h$ and $p_h$ to the left hand side we obtain
\begin{IEEEeqnarray*}{cCcCl"l}
A(y_h, v_h) &+& \langle \beta^2 \lambda^{-1} p_h, v_h \rangle_{L^2(\Sigma_R)} &=& \langle y_0, v_h(0) \rangle_{L^2(\Omega)} & \forall v_h \in \Shp(\meshT_N) \\
A(v_h, p_h) &-& \langle y_h(T), v_h(T) \rangle_{L^2(\Omega)}  &=& - \langle y_\Omega, v_h(T) \rangle_{L^2(\Omega)} & \forall v_h \in \Shp(\meshT_N)
\end{IEEEeqnarray*}
To this end, we define two mass matrices matrices in the basis $\varphi_j$, $j = 1, \ldots, m$ of $\Shp(\meshT_N)$:
\begin{IEEEeqnarray*}{rCl}
	M_{\Sigma, h}[i, j] &\coloneqq& \langle \varphi_j, \varphi_i \rangle_{L^2(\Sigma_R)}, \\
	M_{\Omega, h}[i, j] &\coloneqq& \langle \varphi_j(T), \varphi_i(T) \rangle_{L^2(\Omega)}.
\end{IEEEeqnarray*}
Furthermore, we introduce a matrix with the constant $M^u_{\Sigma, h}[i, j] \coloneqq \beta^2 \lambda^{-1} M_{\Sigma, h}[i, j]$.
By defining $\boldsymbol{x_h} = (\boldsymbol{y_h}, \boldsymbol{p_h})$ as a new coefficient vector corresponding to a function in $\Shp(\meshT_N) \times \Shp(\meshT_N)$, we can bring this together into a single large system:
\begin{IEEEeqnarray*}{cCcCl}
A_h \boldsymbol{y_h} &+& M^u_{\Sigma, h} \boldsymbol{p_h} &=& \boldsymbol{g^\Sigma_h} \\
-M_{\Omega, h} \boldsymbol{y_h} &+& A_h^\tp \boldsymbol{p_h} &=& \boldsymbol{g^\Omega_h}
\end{IEEEeqnarray*}
Hereby we have the definitions
\begin{IEEEeqnarray*}{rCl}
	\boldsymbol{g^\Sigma_h}[i] &\coloneqq& \langle y_0, \varphi_i(0) \rangle_{L^2(\Omega)} \\
	\boldsymbol{g^\Omega_h}[i] &\coloneqq& - \langle y_\Omega, \varphi_i(T) \rangle_{L^2(\Omega)}.
\end{IEEEeqnarray*}
Because $M^u_{\Sigma, h}$ and $M_{\Omega, h}$ are symmetric by definition, we exchange the equations and define
\[
	Y_h \coloneqq \begin{bmatrix}
		-M_{\Omega, h} & A_h^\tp \\
		A_h & M^u_{\Sigma, h}
	\end{bmatrix}
\]
and a combined vector $\boldsymbol{g_h} = (\boldsymbol{g^\Omega_h}, \boldsymbol{g^\Sigma_h})$.
Then $Y_h$ is a symmetric matrix and our problem can be expressed as
\[
	Y_h \boldsymbol{x_h} = \boldsymbol{g_h}.
\]
It can be easily seen to be indefinite, because $M_{\Omega, h}$ and $M^u_{\Sigma, h}$ are mass matrix of boundary integrals.
Therefore, $-M_{\Omega, h}$ is negative and $M^u_{\Sigma, h}$ is positive semidefinite, rendering the system indefinite overall.

We can also make the system a positive semidefinite, but unsymmetrical one by multiplying the first block row with $-1$:
\[
	\begin{bmatrix}
		M_{\Omega, h} & -A_h^\tp \\
		A_h & M^u_{\Sigma, h}
	\end{bmatrix} \boldsymbol{x_h} = \begin{bmatrix}
		- \boldsymbol{g^\Omega_h} \\
		\boldsymbol{g^\Sigma_h}
	\end{bmatrix}.
\]
Being mass matrices, $-K_h$ and $J_h$ are positive semi-definite. For any $\boldsymbol{x_h}$ we have:
\begin{IEEEeqnarray*}{rCl}
	\boldsymbol{x_h}^\tp \begin{bmatrix}
		M_{\Omega, h} & -A_h^\tp \\
		A_h & M^u_{\Sigma, h}
	\end{bmatrix} \boldsymbol{x_h} &=& \boldsymbol{y_h}^\tp M_{\Omega, h} \boldsymbol{y_h} + \boldsymbol{y_h}^\tp (-A_h^\tp) \boldsymbol{p_h} + \boldsymbol{p_h}^\tp A_h \boldsymbol{y_h} + \boldsymbol{p_h}^\tp M^u_{\Sigma, h} \boldsymbol{p_h} \\
	&=& \boldsymbol{y_h}^\tp M_{\Omega, h} \boldsymbol{y_h} + \boldsymbol{p_h}^\tp M^u_{\Sigma, h} \boldsymbol{p_h} \\
	&\geq& 0.
\end{IEEEeqnarray*}

We can proceed analogously for the interior heat source formulation.
Starting from the optimality system, which holds for all $v_h \in \Shp(\meshT_N)$ again,
\begin{IEEEeqnarray*}{c}
\begin{IEEEeqnarraybox}{rCl"rCl}
A(y_h, v_h) &=& \langle \beta u_h, v_h \rangle_{L^2(Q)} & A(v_h, p_h) &=& \langle y_h - y_\Sigma, v_h \rangle_{L^2(\Sigma)}
\end{IEEEeqnarraybox} \\
u_h = - \lambda^{-1} \beta p_h,
\end{IEEEeqnarray*}
we transform into a system of equations given as
\begin{IEEEeqnarray*}{cCcCl"l}
A(y_h, v_h) &+& \langle \beta^2 \lambda^{-1} p_h, v_h \rangle_{L^2(Q)} &=& 0 & \forall v_h \in \Shp(\meshT_N) \\
A(v_h, p_h) &-& \langle y_h, v_h \rangle_{L^2(\Sigma)}  &=& - \langle y_\Sigma, v_h \rangle_{L^2(\Sigma)} & \forall v_h \in \Shp(\meshT_N).
\end{IEEEeqnarray*}
Similarly, we introduce the following matrix
\begin{IEEEeqnarray*}{rCl}
	M_{Q,h}[i, j] &\coloneqq&  \langle \varphi_j, \varphi_i \rangle_{L^2(Q)}, \\
\noalign{\noindent and define the matrix $M^u_{Q,h}[i, j] \coloneqq \beta^2 \lambda^{-1} M_{Q,h}[i,j]$. Furthermore, we introduce the associated right-hand side vectors: \vspace{\jot}}
	\boldsymbol{g^{Q,I}_h}[i] &\coloneqq& 0 \\
	\boldsymbol{g^{\Sigma,I}_h}[i] &\coloneqq& - \langle y_\Sigma, \varphi_i \rangle_{L^2(\Sigma)}.
\end{IEEEeqnarray*}
By exchanging the equations as before, we define the analogous matrix
\[
	Y^I_h \coloneqq \begin{bmatrix}
		-M_{\Sigma,h} & A_h^\tp \\
		A_h & M^u_{Q,h}
	\end{bmatrix}
\]
and the associated right hand side $\boldsymbol{g^I_h} = (\boldsymbol{g^{\Sigma,I}_h}, \boldsymbol{g^{Q,I}_h})$.
As before, $Y^I_h$ is symmetric, but indefinite, and it can be transformed into an unsymmetrical, positive semidefinite system.
The overall equation to solve is then
\[
	Y^I_h \boldsymbol{x_h} = \boldsymbol{g^I_h}.
\]

For both, the interior and the boundary heating discretization, uniqueness and existence of solutions of these combined systems is immediately given by the uniqueness of an optimal control.
This is due to the fact that we just transformed the optimality systems which we know to have a unique solution via equivalence transformations into one large matrix.

Moreover, we can recover the optimal control from the solution $\boldsymbol{x_h}$ of the respective systems.
Using the variational equality, we obtain for the interior heating formulation
\[
	u_h = \frac{1}{\lambda} \beta E_V \boldsymbol{p_h}.
\]
For the boundary optimal control, we have to keep in mind that $u_h$ only holds on $L^2(\Sigma)$:
\[
	u_h = E_\Sigma \frac{1}{\lambda} \beta E_V \boldsymbol{p_h}.
\]
\section{Numerical methods for the control-constrained case}
In the control-constrained case, one has to deal with the optimality system having a projection on its right hand side, for example in the case with boundary controls, one obtains a state equation looking like this:
\[
A(y_h, v_h) = \left\langle \beta \projP_{[u_a, u_b]} \left\{ - \lambda^{-1} \beta p_h \right\}, v_h \right\rangle_{L^2(\Sigma_R)} + \langle y_0, v_h(0) \rangle_{L^2(\Omega)} \quad \text{for all $v_h \in \Shp(\meshT_N)$}.
\]
From a theoretical point of view, this is perfectly acceptable, of course.
However, a projection on an interval cannot be a linear function in general and therefore, this poses a problem for an implementation.
The method described in the previous section is simply not applicable in this case.

In effect, one has to look for numerical methods that can handle such a projection formula.
There is a wide variety of options available here, see \cite[p.\ 134ff.]{Troeltzsch} for some discussion on the matter.

Probably the easiest approach is to use a gradient projection method.
Assuming that a number of controls $u_1, \ldots, u_n$ has already been determined, the step to determine $u_{n+1}$ looks as follows (c.f.\ \cite{Troeltzsch}):
\begin{algorithmbox}[Gradient Projection Method]
\begin{tabular}{@{}ll}
\textit{Input:} & An initial control $u_n$. \\
\textit{Output:} & A control $u_{n+1}$.
\end{tabular}
\end{algorithmbox}
\vspace{-10pt}
\begin{algorithm}
Determine the discretized state $y_n$.\;
Calculate the adjoined state $p_n$.\;
Set the direction of descent to the anti-gradient
\[
	v_n = -J_h'(u_n) = - (\beta p_n + \lambda u_n).
\]\;
\vspace{-15pt}
Determine an optimal step width $s_n$ from
\[
	J_h \left( \projP_{[u_a, u_b]} \left\{ u_n + s_n v_n \right\} \right) = \min_{s > 0} J_h \left( \projP_{[u_a, u_b]} \left\{ u_n + s v_n \right\} \right).
\]\;
\vspace{-15pt}
Set the next control as $u_{n+1} \coloneqq \projP_{[u_a, u_b]} \left\{ u_n + s_n v_n \right\}$.
\end{algorithm}
\vspace{-4pt}
\EndAlgorithmLine
The convergence of this method has been analyzed in e.g.\ \cite{HinzePinnauUlbrich}, however it tends to become rather slow for practical purposes.
Furthermore, given that the projection does not have to be a mesh function, this is difficult to implement unless a control discretization is being used, compare the discussion in the following subsection.
\subsection{Primal-dual active set strategies}
\label{sec:KR-numerics}
Another promising approach is to apply so called primal-dual active set strategies. The works \cite{ItoKunisch-2000}, \cite{ItoKunisch} and \cite{BergouniouxItoKunisch} discuss the origins of these methods. Specifically for the parabolic case considered here, the method described in \cite{KunischRoesch} can be applied.
Roughly spoken, these methods maintain some active sets where the box restrictions for $u$ are enforced. Outside of those sets, the unrestricted system as presented in the previous section is applied.

We want to present how the method treated in \cite{KunischRoesch} can be applied for our discontinuous Galerkin method. For this purpose, we let $E$ be the domain that the optimal control operates on, i.e.\ $E = \Sigma$ for the boundary heating scenario and $E = Q$ for the interior one. Furthermore, let $Y$ be the state space and $U$ be the control space, i.e.\ $U = L^2(E)$.
The problem we want to operate on can then be expressed as
\begin{problem}
\label{prb:KR-optimal-control}
\begin{IEEEeqnarray*}{c}
\min J(y, u) = \frac{1}{2} \| y - y_d \|^2_Y + \frac{\lambda}{2} \| u \|_U^2 \\
\noalign{\noindent subject to\vspace{\jot}}
\begin{IEEEeqnarraybox}{rCl}
y &=& S_h u + q
\end{IEEEeqnarraybox} \\
\noalign{\noindent and\vspace{\jot}}
u_a \leq u(x, t) \leq u_b\quad \text{a.e.\ in $E$}.
\end{IEEEeqnarray*}
\end{problem}
Hereby let $q, y_d \in Y$, $\lambda > 0$ and $-\infty \leq u_a < u_b \leq \infty$ hold, where at least $u_a$ or $u_b$ are finite.
It is assumed in \cite{KunischRoesch} that $S_h$ is linear and compact, but given the finite dimension of $\Shp(\meshT_N)$, this is ensured to be the case.
Moreover, we assume here that the previously used $\beta$ is constant and equal to one.
\begin{theorem}
\label{thm:KR-Lagrange-mult}
Let $c > 0$. A control $\bar{u}_h$ and $\bar{y}_h$ is optimal for \cref{prb:KR-optimal-control} if and only if there exists a $(\bar{p}_h, \bar{\mu}_h) \in U \times U$ such that the following two conditions hold for $(\bar{u}_h, \bar{y}_h, \bar{p}_h, \bar{\mu}_h)$:
\begin{IEEEeqnarray*}{rCl}
\bar{y}_h &=& S_h \bar{u}_h + q, \\
\bar{p}_h &=& S_h^*(\bar{y} - yd), \\
-\bar{p}_h &=& \lambda \bar{u}_h + \bar{\mu}_h
\end{IEEEeqnarray*}
and
\begin{IEEEeqnarray*}{rCl}
\bar{\mu}_h &=& c \left( \bar{u}_h + \frac{1}{c} \bar{\mu}_h - \projP_{[u_a, u_b]} \left( \bar{u}_h + \frac{1}{c} \bar{\mu}_h \right) \right).
\end{IEEEeqnarray*}
\end{theorem}
\begin{proof}
See \cite{KunischRoesch} for the proof of this statement.
\end{proof}
Next, we define the following sets, the so called active and inactive sets for the optimal solution:
\begin{IEEEeqnarray*}{rCl}
	\bar{A}^+ &=& \left\{ x \in E \midcolon \bar{u}_h(x) = u_b \right\}, \\
	\bar{A}^- &=& \left\{ x \in E \midcolon \bar{u}_h(x) = u_a \right\}, \\
	\bar{I} &=& \left\{ x \in E \midcolon u_a < \bar{u}_h(x) < u_b \right\}.
\end{IEEEeqnarray*}
The algorithm will proceed in an iterative fashion, determining pairs of controls and Lagrange multipliers $(u_n, \mu_n) \in U \times U$.
Based on the multiplier rule of \cref{thm:KR-Lagrange-mult}, we introduce for $(u_{n-1}, \mu_{n-1}) \in U \times U$ the following active sets for the next iterate:
\begin{equation}
\label{eq:KR-active-sets}
\begin{IEEEeqnarraybox}[][c]{rCl}
	A^+_n &=& \left\{ x \in E \midcolon u_{n-1}(x) + \frac{\mu_{n-1}(x)}{c} > u_b \right\}, \\
	A^-_n &=& \left\{ x \in E \midcolon u_{n-1}(x) + \frac{\mu_{n-1}(x)}{c} < u_a \right\}, \\
	I_n &=& E \setminus ( A^+_n \cup A^-_n ).
\end{IEEEeqnarraybox}
\end{equation}
One can conclude from the Lagrange multiplier rule in \cref{thm:KR-Lagrange-mult}, that for the optimal control one has $\bar{\mu} \geq 0$ on $\bar{A}^+$ and $\bar{\mu} \leq 0$ on $\bar{A}^-$. This motivates the choice of the active set update strategies in the algorithm.
Finally, we can describe the algorithm as given in \cite{KunischRoesch}:
\needspace{20\baselineskip}
\begin{algorithmbox}[Primal-dual Active Set Strategy]
\begin{tabular}{@{}ll}
\textit{Input:} & An initial control $u_0$ and a multiplier $\mu_0$. \\
\textit{Output:} & A control tuple $(\bar{u}_h, \bar{y}_h, \bar{p}_h, \bar{\mu}_h)$.
\end{tabular}
\end{algorithmbox}
\vspace{-10pt}
\begin{algorithm}
Set $n = 1$.\;
Determine the active and inactive sets for $(u_n, \mu_n)$ as given by \cref{eq:KR-active-sets}.\label{alg:KR-step-1}\;
\eIf{$n \geq 2$ and $A_n^+ = A_{n-1}^+$, $A_n^- = A_{n-1}^-$, and $I_n = I_{n-1}$}{
	\Return $(u_{n-1}, y_{n-1}, p_{n-1}, \mu_{n-1})$.
}{
	Find $(y_n, p_n) \in Y \times U$ such that
	\begin{IEEEeqnarray*}{rCl}
		y_n &=& S_h u_n + q, \\
		p_n &=& S_h^* (y_n - y_d),
	\end{IEEEeqnarray*}
	where
	\[
		u_n(x) = \begin{cases}
			u_b & \text{if } x \in A_n^+, \\
			u_a & \text{if } x \in A_n^-, \\
			- \frac{p_n}{\lambda} & \text{if } x \in I_n.
		\end{cases}
	\]\;
	Set $\mu_n = p_n + \lambda u_n$, $n \leftarrow n + 1$ and goto \cref{alg:KR-step-1}.
}
\end{algorithm}
\vspace{-4pt}
\EndAlgorithmLine
It can be easily seen that if the algorithm terminates, one has
\begin{IEEEeqnarray*}{rCl"l}
	\lambda_n &=& 0 & \text{on $I_n$}, \\
	\lambda_n &>& 0 & \text{on $A_n^+$}, \\
	\lambda_n &<& 0 & \text{on $A_n^-$}.	 
\end{IEEEeqnarray*}
and therefore the solution fulfills the optimality conditions of \cref{thm:KR-Lagrange-mult}, see \cite[Theorem 1]{KunischRoesch}.
For a discussion of the stopping behavior of the algorithm and convergence analysis, we refer to \cite{KunischRoesch} again.

Now, the issue how to find $(y_n, p_n) \in Y \times U$ remains. Previously, we could just bring the control  $u_n$ from the right hand side into the linear equation system on the left as $- \lambda^{-1} \beta p_h \in \Shp(\meshT_N)$.
However, the constrained control $u_n \notin \Shp(\meshT_N)$.
It would be very difficult to approach storing both, the active sets and the control iterate $u_n$ for an implementation.
In general, because the optimal control can have a very difficult to handle form, working with the control space $L^2(E)$ is not practicable. Normally, one introduces a control discretization at this point, limiting the control space to a discretized subspace that one can handle better.

It should be noted that restricting the control to a subspace does not invalidate the optimality conditions and uniqueness of the discrete problem.
All proofs related to these work just as well with the choice $U_{ad} \subset \Shp(\meshT_N)$.
Obviously, such a choice would be still closed and because a convex combination of two functions of $\Shp(\meshT_N)$ still leads to a function in $\Shp(\meshT_N)$, convexity of $U_{ad}$ is not a problem, either.
However, this leads to a new discrete optimal solution for the fully discretized problem that is difficult to relate to the control being searched in the non-discretized space.
For elliptic problems, \cite{CasasTroeltzsch} proves results for constant and linear discretization approaches of the control space.
The previously mentioned work \cite{MeidnerVexler-I} was extended by its authors to cover these discretized control spaces in \cite{MeidnerVexler-II}.
It might be possible to extend or use the ideas in \cite{MeidnerVexler-II} for application with the discontinuous Galerkin method as presented here.

For the time being, we will just discuss how our numerical approach will look like without proving any error estimates.
In the next chapter we present numerical results for this method.
On a discrete level, we have now the choices $Y = \Shp(\meshT_N)$ and $U$ is chosen either as $\Shp(\meshT_N)$ (case $E = Q$) or the boundary projection of $\Shp(\meshT_N)$ (case $E = \Sigma$).
This means we have to solve the system
\begin{IEEEeqnarray*}{rCl"l}
A(y_h, v_h) &=& \langle u_n, v_h \rangle_{U} + \langle \tilde{q}, v_h \rangle_{\tilde{Y}}& \forall v_h \in \Shp(\meshT_N), \\
A(v_h, p_h) &=& \langle y_h - y_d, v_h \rangle_{L^2(Y)} & \forall v_h \in \Shp(\meshT_N).
\end{IEEEeqnarray*}
Hereby we denote with $\tilde{q} \in \tilde{Y}$ the function that the discontinuous Galerkin method solves to $q$, with $\tilde{Y}$ being chosen accordingly. For example in the case of boundary controls, this would be our initial condition and $\tilde{Y} =L^2(\Omega) \times \{ 0\}$.

We now use that $A_n^+ \cup A_n^- \cup I_n = E$ by definition of these sets.
Assuming we have a nodal basis of $\Shp(\meshT_N)$ that is being used, we can separate $u_n$ based on these sets. Therefore let $u_n^+$, $u_n^-$ and $u_n^I$ be functions in $\Shp(\meshT_N)$ with the same coefficients as $u_n$ if the associated basis function has its node in $A_n^+$, $A_n^-$ or $I_n$, respectively and otherwise a zero coefficient.
In other words, if $\Shp(\meshT_N) = \spn \{ \varphi_j \}_{j=1}^m$, where $\varphi_j$ are nodal basis functions associated with nodes $\{ s_j \}_{j=1}^m$, so that $\varphi_j(s_i) = \delta_{ij}$ for all $i,j = 1, \ldots, m$, then
\begin{IEEEeqnarray*}{rCl"l}
	u_n^+(x, t) &=& \sum_{j=1}^m \boldsymbol{u_n^+}[j] \varphi_j(x, t) 
\end{IEEEeqnarray*}
with
\[
	\boldsymbol{u_n^+}[j] = \begin{cases}
		\boldsymbol{u}[j] & \text{if $s_j \in A_n^+$}, \\
		0 & \text{else}.
	\end{cases}
\]
The functions $u_n^-$ and $u_n^I$ are defined analogously.
With this definition, $u_n = u_n^+ + u_n^- + u_n^I$.
That means, we can interpret the first equation as
\begin{IEEEeqnarray*}{rCl}
A(y_h, v_h) &=& \langle u_n, v_h \rangle_{U} + \langle \tilde{q}, v_h \rangle_{\tilde{Y}} \\
&=& \langle u_n^-, v_h \rangle_{U} + \langle u_n^+, v_h \rangle_{U} + \langle u_n^I, v_h \rangle_{U} + \langle \tilde{q}, v_h \rangle_{\tilde{Y}}.
\end{IEEEeqnarray*}
Motivated by the choice of $u_n$ in the algorithm, one makes the following approach for the discrete coefficients:
\[
	\boldsymbol{u_n}[j] = \begin{cases}
		u_b & \text{for $s_j \in A_n^+$}, \\
	 	u_a & \text{for $s_j \in A_n^-$}, \\
	 	- \lambda^{-1} \boldsymbol{p_n}[j] & \text{for $s_j \in I_n$}.
	\end{cases}
\]
Proving that the optimal discretized control is of this form can be done by adapting the steps necessary for proving \cref{thm:KR-Lagrange-mult} on the discrete level, where one can associate a multiplier $\mu$ with each basis function, effectively.  

This means we can consider $u_n^+$ and $u_n^-$ as functions for the right hand side and only have to bring $u_n^I = - \lambda^{-1} p_n^I$ to the other side.
From here on out, we can proceed as in the last section.
However, the term $\langle u_n^I, v_h \rangle_U$ leads to a mass matrix block that is not symmetrical: Only functions that have their nodes in $I_n$ are evaluated against all functions in $\Shp(\meshT_N)$. This means that we have to solve the linear equation without profiting of any speedups that might arise from using methods specialized for symmetrical matrices.
\end{document}