\documentclass[../thesis.tex]{subfiles}

\begin{document}
\chapter{Numerical treatment of the control problem}
In \cref{sec:Disc-Control-Problem} we derived optimality conditions and convergence for an approximation to the optimal control heating problem for both, boundary and interior heat sources.
We now want to discuss numerical methods for solving the formulations established.
For now, we consider the case with no restrictions made to the control, i.e.\ $u_a = -\infty$ and $u_b = \infty$. In that setup, the projection formula becomes a simple equation: $u_h = - \lambda^{-1} \beta p_h$.
A key advantage of the discontinuous Galerkin approach presented here is that we can immediately assemble a linear equation system for this case and solve the already discretized system with a common equation solver.
\section{Establishing a linear optimality system in the case with no control restrictions}
In this section, we assume $u_a = -\infty$ and $u_b = \infty$, as well as $\beta$ being a constant with $\beta > 0$. 
For the boundary heat system we obtain under this setup an optimality system of
\begin{IEEEeqnarray*}{c}
\begin{IEEEeqnarraybox}{rCl"rCl}
A(y_h, v_h) &=& \langle \beta u_h, v_h \rangle_{L^2(\Sigma_R)} + \langle y_0, v_h(0) \rangle_{L^2(\Omega)} & A(v_h, p_h) &=& \langle y_h(T) - y_\Omega, v_h(T) \rangle_{L^2(\Omega)}
\end{IEEEeqnarraybox} \\
u_h = - \lambda^{-1} \beta p_h.
\end{IEEEeqnarray*}
holding for all $v_h \in \Shp(\meshT_N)$.
By substituting $u_h$ with the representation given by the variational equality, we obtain the following system:
\begin{IEEEeqnarray*}{rCl"l}
A(y_h, v_h) &=& - \langle \beta^2 \lambda^{-1} p_h, v_h \rangle_{L^2(\Sigma_R)} + \langle y_0, v_h(0) \rangle_{L^2(\Omega)}& \forall v_h \in \Shp(\meshT_N) \\
A(v_h, p_h) &=& \langle y_h(T) - y_\Omega, v_h(T) \rangle_{L^2(\Omega)} & \forall v_h \in \Shp(\meshT_N)
\end{IEEEeqnarray*}
If we bring the terms depending on $y_h$ and $p_h$ to the left hand side we obtain
\begin{IEEEeqnarray*}{cCcCl"l}
A(y_h, v_h) &+& \langle \beta^2 \lambda^{-1} p_h, v_h \rangle_{L^2(\Sigma_R)} &=& \langle y_0, v_h(0) \rangle_{L^2(\Omega)} & \forall v_h \in \Shp(\meshT_N) \\
A(v_h, p_h) &-& \langle y_h(T), v_h(T) \rangle_{L^2(\Omega)}  &=& - \langle y_\Omega, v_h(T) \rangle_{L^2(\Omega)} & \forall v_h \in \Shp(\meshT_N)
\end{IEEEeqnarray*}
To this end, we define two mass matrices matrices in the basis $\varphi_j$, $j = 1, \ldots, m$ of $\Shp(\meshT_N)$:
\begin{IEEEeqnarray*}{rCl}
	M_{\Sigma, h}[i, j] &\coloneqq& \langle \varphi_j, \varphi_i \rangle_{L^2(\Sigma_R)}, \\
	M_{\Omega, h}[i, j] &\coloneqq& \langle \varphi_j(T), \varphi_i(T) \rangle_{L^2(\Omega)}.
\end{IEEEeqnarray*}
Furthermore, we introduce a matrix with the constant $M^u_{\Sigma, h}[i, j] \coloneqq \beta^2 \lambda^{-1} M_{\Sigma, h}[i, j]$.
By defining $\boldsymbol{x_h} = (\boldsymbol{y_h}, \boldsymbol{p_h})$ as a new coefficient vector corresponding to a function in $\Shp(\meshT_N) \times \Shp(\meshT_N)$, we can bring this together into a single large system:
\begin{IEEEeqnarray*}{cCcCl}
A_h \boldsymbol{y_h} &+& M^u_{\Sigma, h} \boldsymbol{p_h} &=& \boldsymbol{g^\Sigma_h} \\
-M_{\Omega, h} \boldsymbol{y_h} &+& A_h^\tp \boldsymbol{p_h} &=& \boldsymbol{g^\Omega_h}
\end{IEEEeqnarray*}
Hereby we have the definitions
\begin{IEEEeqnarray*}{rCl}
	\boldsymbol{g^\Sigma_h}[i] &\coloneqq& \langle y_0, \varphi_i(0) \rangle_{L^2(\Omega)} \\
	\boldsymbol{g^\Omega_h}[i] &\coloneqq& - \langle y_\Omega, \varphi_i(T) \rangle_{L^2(\Omega)}.
\end{IEEEeqnarray*}
Because $M^u_{\Sigma, h}$ and $M_{\Omega, h}$ are symmetric by definition, we exchange the equations and define
\[
	Y_h \coloneqq \begin{bmatrix}
		-M_{\Omega, h} & A_h^\tp \\
		A_h & M^u_{\Sigma, h}
	\end{bmatrix}
\]
and a combined vector $\boldsymbol{g_h} = (\boldsymbol{g^\Omega_h}, \boldsymbol{g^\Sigma_h})$.
Then $Y_h$ is a symmetric matrix and our problem can be expressed as
\[
	Y_h \boldsymbol{x_h} = \boldsymbol{g_h}.
\]
It can be easily seen to be indefinite, because $M_{\Omega, h}$ and $M^u_{\Sigma, h}$ are mass matrix of boundary integrals.
Therefore, $-M_{\Omega, h}$ is negative and $M^u_{\Sigma, h}$ is positive semidefinite, rendering the system indefinite overall.

We can also make the system a positive semidefinite, but unsymmetrical one by multiplying the first block row with $-1$:
\[
	\begin{bmatrix}
		M_{\Omega, h} & -A_h^\tp \\
		A_h & M^u_{\Sigma, h}
	\end{bmatrix} \boldsymbol{x_h} = \begin{bmatrix}
		- \boldsymbol{g^\Omega_h} \\
		\boldsymbol{g^\Sigma_h}
	\end{bmatrix}.
\]
Being mass matrices, $-K_h$ and $J_h$ are positive semi-definite. For any $\boldsymbol{x_h}$ we have:
\begin{IEEEeqnarray*}{rCl}
	\boldsymbol{x_h}^\tp \begin{bmatrix}
		M_{\Omega, h} & -A_h^\tp \\
		A_h & M^u_{\Sigma, h}
	\end{bmatrix} \boldsymbol{x_h} &=& \boldsymbol{y_h}^\tp M_{\Omega, h} \boldsymbol{y_h} + \boldsymbol{y_h}^\tp (-A_h^\tp) \boldsymbol{p_h} + \boldsymbol{p_h}^\tp A_h \boldsymbol{y_h} + \boldsymbol{p_h}^\tp M^u_{\Sigma, h} \boldsymbol{p_h} \\
	&=& \boldsymbol{y_h}^\tp M_{\Omega, h} \boldsymbol{y_h} + \boldsymbol{p_h}^\tp M^u_{\Sigma, h} \boldsymbol{p_h} \\
	&\geq& 0.
\end{IEEEeqnarray*}

We can proceed analogously for the interior heat source formulation.
Starting from the optimality system, which holds for all $v_h \in \Shp(\meshT_N)$ again,
\begin{IEEEeqnarray*}{c}
\begin{IEEEeqnarraybox}{rCl"rCl}
A(y_h, v_h) &=& \langle \beta u_h, v_h \rangle_{L^2(Q)} & A(v_h, p_h) &=& \langle y_h - y_\Sigma, v_h \rangle_{L^2(\Sigma)}
\end{IEEEeqnarraybox} \\
u_h = - \lambda^{-1} \beta p_h,
\end{IEEEeqnarray*}
we transform into a system of equations given as
\begin{IEEEeqnarray*}{cCcCl"l}
A(y_h, v_h) &+& \langle \beta^2 \lambda^{-1} p_h, v_h \rangle_{L^2(Q)} &=& 0 & \forall v_h \in \Shp(\meshT_N) \\
A(v_h, p_h) &-& \langle y_h, v_h \rangle_{L^2(\Sigma)}  &=& - \langle y_\Sigma, v_h \rangle_{L^2(\Sigma)} & \forall v_h \in \Shp(\meshT_N).
\end{IEEEeqnarray*}
Similarly, we introduce the following matrix
\begin{IEEEeqnarray*}{rCl}
	M_{Q,h}[i, j] &\coloneqq&  \langle \varphi_j, \varphi_i \rangle_{L^2(Q)}, \\
\noalign{\noindent and define the matrix $M^u_{Q,h}[i, j] \coloneqq \beta^2 \lambda^{-1} M_{Q,h}[i,j]$. Furthermore, we introduce the associated right-hand side vectors: \vspace{\jot}}
	\boldsymbol{g^{Q,I}_h}[i] &\coloneqq& 0 \\
	\boldsymbol{g^{\Sigma,I}_h}[i] &\coloneqq& - \langle y_\Sigma, \varphi_i \rangle_{L^2(\Sigma)}.
\end{IEEEeqnarray*}
By exchanging the equations as before, we define the analogous matrix
\[
	Y^I_h \coloneqq \begin{bmatrix}
		-M_{\Sigma,h} & A_h^\tp \\
		A_h & M^u_{Q,h}
	\end{bmatrix}
\]
and the associated right hand side $\boldsymbol{g^I_h} = (\boldsymbol{g^{\Sigma,I}_h}, \boldsymbol{g^{Q,I}_h})$.
As before, $Y^I_h$ is symmetric, but indefinite, and it can be transformed into an unsymmetrical, positive semidefinite system.
The overall equation to solve is then
\[
	Y^I_h \boldsymbol{x_h} = \boldsymbol{g^I_h}.
\]

For both, the interior and the boundary heating discretization, uniqueness and existence of solutions of these combined systems is immediately given by the uniqueness of an optimal control.
This is due to the fact that we just transformed the optimality systems which we know to have a unique solution via equivalence transformations into one large matrix.

Moreover, we can recover the optimal control from the solution $\boldsymbol{x_h}$ of the respective systems.
Using the variational equality, we obtain for the interior heating formulation
\[
	u_h = \frac{1}{\lambda} \beta E_V \boldsymbol{p_h}.
\]
For the boundary optimal control, we have to keep in mind that $u_h$ only holds on $L^2(\Sigma)$:
\[
	u_h = E_\Sigma \frac{1}{\lambda} \beta E_V \boldsymbol{p_h}.
\]
\end{document}