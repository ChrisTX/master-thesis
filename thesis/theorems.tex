% -*- root: lecture-notes.tex -*-

\newcounter{dummythm}
\numberwithin{dummythm}{chapter}
\mdfdefinestyle{theoremframing}{skipabove=\topskip,skipbelow=\topskip,nobreak=true,innerleftmargin=7pt,innerrightmargin=7pt}

\theoremseparator{.}
\theoremstyle{plain}
\newmdtheoremenv[style=theoremframing,backgroundcolor=LimeGreen!100!white!20,linecolor=LimeGreen]{theorem}[dummythm]{Theorem}
\crefname{theorem}{Theorem}{Theorems}
\newmdtheoremenv[style=theoremframing,backgroundcolor=LimeGreen!100!white!10,linecolor=LimeGreen]{lemma}[dummythm]{Lemma}
\crefname{lemma}{Lemma}{Lemmata}
\newmdtheoremenv[style=theoremframing,backgroundcolor=LimeGreen!100!white!10,linecolor=LimeGreen]{corollary}[dummythm]{Corollary}
\crefname{corollary}{Corollary}{Corollaries}
\newmdtheoremenv[style=theoremframing,backgroundcolor=LimeGreen!100!white!10,linecolor=LimeGreen]{proposition}[dummythm]{Proposition}
\crefname{proposition}{Proposition}{Propositions}
\theorembodyfont{\normalfont}
\newmdtheoremenv[style=theoremframing,backgroundcolor=cyan!100!white!10,linecolor=cyan!100!white!90]{definition}[dummythm]{Definition}
\crefname{definition}{Definition}{Definitions}

\theoremstyle{nonumberplain}
\newtheorem{example}{Example}
\crefname{example}{Example}{Examples}
\theoremheaderfont{\itshape}
\newtheorem{remark}{Remark}
\crefname{remark}{Remark}{Remarks}

\theoremstyle{nonumberplain}
\theoremheaderfont{\normalfont \scshape}
\theorembodyfont{\upshape}
\theoremseparator{.}
\theoremsymbol{\rule{1ex}{1ex}}
\newtheorem{proof}{Proof}
\crefname{proof}{Proof}{Proofs}